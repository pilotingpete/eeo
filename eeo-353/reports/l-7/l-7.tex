\documentclass{article}
%\usepackage{blindtext}
\usepackage[letterpaper, total={6.5in, 9in}]{geometry}
%\usepackage{tabularray}
%\usepackage{hyperref}
%\usepackage{xcolor}
\usepackage{amsmath}
\usepackage{siunitx}
\usepackage{graphicx}
\graphicspath{ {./img/} }
\usepackage{pdfpages}
\usepackage{float}
\usepackage[normalem]{ulem}


\begin{document}
	
\begin{titlepage}
	\centering
	\includegraphics[width=0.45\textwidth]{sbu}\par\vspace{1cm}
	{\LARGE \textsc{EEO335}\par}
	\vspace{1cm}
	{\Large \textsc{Spring 2024}\par}
	\vspace{1.5cm}
	{\huge\bfseries Radio-Frequency Communications Project - ABET\par}
	\vspace{2cm}
	{\Large\itshape Pete Mills\\ID: 115009163\par}
	\vfill
	Professor\par
	Gianluigi \textsc{De Geronimo}

	\vfill

% Bottom of the page
	{\large \today\par}
\end{titlepage}

% Include the original assignment
	\newcommand{\assName}{Assignment 7-1.pdf}

	\includepdf[pages=-,pagecommand=\section*{Copy of Original Assignment}]{\assName}
	%\includepdf[pages=2-,pagecommand={}]{\assName}
	
	%\listoffigures
	
% Begin report

	\section*{Overview}
	
	 In this lab we design and built a frequency modulation transmitter with front end amplifier. The input signal is wirelesly transmitted to an FM radio receiver tuned near \SI{100}{\mHz}. The simulation and experiment are quite close, but the transmit frequency differs by $\approx\SI{6}{\MHz}$. When considering temperature effects, part tolerance, commercial FM receiver frequency display, and other factors, the resulting transmitter is acceptably close in performance to the ideal simulation.
	
	\section{Using the simulator, design and simulate a FM transmitter in the 90-100 MHz frequency band
composed of one signal amplifier circuit followed by one oscillator circuit.}

	\begin{figure}[H]
	    \centering
	    \includegraphics[width=0.7\textwidth]{s1}
	    \caption{An FM transmitter circuit was designed that meets the target performance and within the design constraints. Simulation reveals the circuit has a resonant frequency of $\approx\SI{101}{\MHz}$}.
	\end{figure}

	%\begin{figure}[H]
	 %   \centering
	 %   \includegraphics[width=0.7\textwidth]{calc}
	   % \caption{An air-core inductor was produced using sterling silver wire from my watchmaking supplies. The inductor has 5 turns of 21 gauge wire.}
	%\end{figure}
	
	\section{Prepare an experimental plan to demonstrate your transmitter}
	
	\begin{enumerate}
		\item Build the circuit from the simulation on a breadboard.
		\item Connect an audio source to the input of the amplifier at C2.
		\item Play a distinctive song on the audio source.
		\item Turn ON the FM transmitter and slowly sweep the tuning dial from \SIrange{90}{100}{\MHz}
		\item When the song transmitted is heard, stop tuning and read the tuner frequency.
	\end{enumerate}
	
	\section{Build the circuit at (1) and experimentally demonstrate the transmission of a song generated from the
jack line of your cell phone or your desktop/laptop into the circuit at(1) to a FM receiver (e.g. car)}
	
	\begin{figure}[H]
	    \centering
	    \includegraphics[width=0.7\textwidth]{song}
	    \caption{An FM transmitter circuit transmitting classical music to an FM receiver.}
	\end{figure}
	
	\begin{figure}[H]
	    \centering
	    \includegraphics[width=0.7\textwidth]{freq}
	    \caption{FM receiver tuned to $\approx\SI{94}{\MHz}$}.
	\end{figure}

\end{document}
