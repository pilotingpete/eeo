\documentclass[10pt,landscape]{article}
\usepackage{multicol}
%\usepackage{calc}
\usepackage{amssymb}
\usepackage{ifthen}
\usepackage[landscape]{geometry}
\usepackage{graphicx}
%\usepackage{verbatim}
\usepackage{siunitx}
\usepackage{amsmath}



% This sets page margins to .5 inch if using letter paper, and to 1cm
% if using A4 paper. (This probably isn't strictly necessary.)
% If using another size paper, use default 1cm margins.
\ifthenelse{\lengthtest { \paperwidth = 11in}}
	{ \geometry{top=.5in,left=.5in,right=.5in,bottom=.5in} }
	{\ifthenelse{ \lengthtest{ \paperwidth = 297mm}}
		{\geometry{top=1cm,left=1cm,right=1cm,bottom=1cm} }
		{\geometry{top=1cm,left=1cm,right=1cm,bottom=1cm} }
	}

% Turn off header and footer
\pagestyle{empty}
 

% Redefine section commands to use less space
\makeatletter
\renewcommand{\section}{\@startsection{section}{1}{0mm}%
                                {-1ex plus -.5ex minus -.2ex}%
                                {0.5ex plus .2ex}%x
                                {\normalfont\large\bfseries}}
\renewcommand{\subsection}{\@startsection{subsection}{2}{0mm}%
                                {-1explus -.5ex minus -.2ex}%
                                {0.5ex plus .2ex}%
                                {\normalfont\normalsize\bfseries}}
\renewcommand{\subsubsection}{\@startsection{subsubsection}{3}{0mm}%
                                {-1ex plus -.5ex minus -.2ex}%
                                {1ex plus .2ex}%
                                {\normalfont\small\bfseries}}
\makeatother

% Define BibTeX command
\def\BibTeX{{\rm B\kern-.05em{\sc i\kern-.025em b}\kern-.08em
    T\kern-.1667em\lower.7ex\hbox{E}\kern-.125emX}}

% Don't print section numbers
\setcounter{secnumdepth}{0}


\setlength{\parindent}{0pt}
\setlength{\parskip}{0pt plus 0.5ex}


% -----------------------------------------------------------------------

\begin{document}

\raggedright
\footnotesize
\begin{multicols}{3}


% multicol parameters
% These lengths are set only within the two main columns
%\setlength{\columnseprule}{0.25pt}
\setlength{\premulticols}{1pt}
\setlength{\postmulticols}{1pt}
\setlength{\multicolsep}{1pt}
\setlength{\columnsep}{2pt}

% Page title
\begin{center}
     \Large{\textbf{EE Formula Sheet}} \\
\end{center}

\newlength{\MyLen}

%%%%%%%%%%%%%%%%%%_ Begin New Section _%%%%%%%%%%%%%%%%%%%

\subsection{Constants}
\begin{align*}
q &= 1.602 \times 10^{-19}\ \mathrm{C} \\
m_e &= 9.109 \times 10^{-31}\ \mathrm{kg} \\
h &= 6.626 \times 10^{-34}\ \mathrm{J}\cdot\mathrm{s} \\
k &= 1.381 \times 10^{-23}\ \mathrm{J/K} \\
\varepsilon_0 &= 8.854 \times 10^{-12}\ \mathrm{C}^2/(\mathrm{N}\cdot\mathrm{m}^2) \\
c &= 3.00 \times 10^8\ \mathrm{m/s}\\
e &\approx 1.6 \times 10^{-19}\ \text{C}\\
kT &\approx 0.026 \mathrm{eV} \mathrm{at} T = 300 \text{K}\\
\end{align*}

\subsection{Formulae}
$kT_{temp} = 0.026 (\frac{temp}{300})$, kT at a temperature temp\\
$\sigma = en\mu_n + ep\mu_p$, Conduction\\
$ n_i^2 = n_0 p_0$, concentration at equilibrium\\
$ n_i^2 = n_c n_v e^{-E_g/kT}$, \\
$n_i^2 \propto T^3 e^{-E_g/kT}$, proportionality ratio\\
$n_i^2 at {500} =(\frac{500}{300})^3 e^{-E_g/kT at 500} e^{E_g/kT at 300}$, proportional temp\\
$E = \frac{hc}{\lambda}$, energy of photon\\
$E_g = E_c - E_v$, Energy band gap\\
$f(E) = \frac{1}{1 + e^{\frac{E - E_f}{kT}}}$,  Fermi-Dirac Distribution Function\\
$n = N_c \cdot e^{-\frac{E_c - E_f}{kT}}$, Electron carrier concentration\\
$p = N_v \cdot e^{-\frac{E_f - E_v}{kT}}$, Hole carrier concentration\\
$J_d = q \cdot n \cdot \mu_n \cdot E$, Drift Current\\
$J_n = q \cdot D_n \cdot \frac{dn}{dx}$, Diffusion Current\\
$E_g = E_c - E_v$, Energy-Band Gap (Eg)

$\frac{1}{m^*} = \frac{1}{m_l} + \frac{1}{m_t}$, Electron and Hole Effective Mass\\
$q = 1.602 \times 10^{-19}\ \mathrm{C}$, Charge of an Electron\\
$n = N_c \cdot e^{-\frac{E_c - E_f}{kT}}$, Electron Carrier Concentration\\
$p = N_v \cdot e^{-\frac{E_f - E_v}{kT}}$, Hole Carrier Concentration\\
$J_n = q \cdot n \cdot \mu_n \cdot E$, Drift Current Density for Electrons\\
$J_p = q \cdot p \cdot \mu_p \cdot E$, Drift Current Density for Holes\\
$J_n = q \cdot D_n \cdot \frac{dn}{dx}$, Diffusion Current Density for Electrons\\
$J_p = q \cdot D_p \cdot \frac{dp}{dx}$, Diffusion Current Density for Holes\\
$N_c = 2 \left(\frac{2\pi m_e kT}{h^2}\right)^{3/2}$, Density of States in the Conduction Band (Nc)\\
$N_v = 2 \left(\frac{2\pi m_h kT}{h^2}\right)^{3/2}$, Density of States in the Valence Band (Nv)\\
$P_0 = n_i e^{\frac{E_{fi} - E_f}{kT}}$\\
$P_0 = \frac{N_A - N_D}{2} + \sqrt{(\frac{N_A - N_D}{2})^2 + n_i^2}$\\
$N_0 = \frac{N_D - N_A}{2} + \sqrt{(\frac{N_D - N_A}{2})^2 + n_i^2}$\\
$f_F(E) = \frac{1}{1 + e^{\frac{E - E_f}{kT}}}$, Fermi-Dirac Distribution Function\\
$f_F(E) = e^{\frac{-(E - E_f)}{kT})}$, Boltzman Approximation when $E-E_F >> kT$\\
$\mu_n = \frac{q \cdot \tau_n}{m^*}$, Electron Mobility\\
$\mu_p = \frac{q \cdot \tau_p}{m^*}$, Hole Mobility\\
$G = \alpha \cdot I$, Generation Rate of Electron-Hole Pairs\\
$R = B \cdot np - A \cdot n_i^2$, Recombination Rate\\
$\frac{\partial n}{\partial t} + \nabla \cdot \mathbf{J}_n = G - R$, Continuity Equation for Electron Current\\
$\frac{\partial p}{\partial t} - \nabla \cdot \mathbf{J}_p = G - R$, Continuity Equation for Hole Current\\
$P_0+N_D=n_0+N_A$, Charge neutrality\\
$J_{drf} = e n \mu_n E + e p \mu_p E = \sigma E$, Total Drift\\
$I = A J_{drf}$, Current\\
$E = Volt/Len$, Electric Field\\
$V_{dn} = \mu_nE$, Drift velocity for electrons\\
$V_{dp} = \mu_pE$, Drift velocity for holes\\

%%%%%%%%%%%%%%%%%%_ Begin New Section _%%%%%%%%%%%%%%%%%%%
\section{\S EEO 311}

$A_v = \frac{V_{out}}{V_{in}}$, Voltage Gain Formula\\
$\lvert A_v \rvert = gm R_D $, \\
$A_v = \frac{R_D}{1/gm} $, \\
$V_{ov} = V_{bias} - V_{th}$, Overdrive Voltage\\
$g_m = K'_n \frac{W}{L} V_{ov}$, transconductance parameter\\
$g_m = \frac{2I_D}{V_{ov}}$\\
$I_D = \frac{1}{2} gm V_{ov}$, Drain Current Equation\\
$I_D = \frac{1}{2} K'_n \frac{W}{L} V_{ov}^2$, Drain Current Equation\\
$I_D = gm \cdot V_{gs}$, Drain current\\
$I_D = \frac{V_{DD}}{R_D}$, Drain current\\
$g_m = \frac{\Delta I_D}{\Delta V_{GS}} = K'_n \frac{W}{L} V_{ov}$,\\
$V_{DD} = V_{DS} + I_D \cdot R_D$, Load line equation\\
$V_{DD} = V_{DS} + I_D \cdot R_D$, Load line equation\\
$r_o = \frac{V_A}{I_D}$, Early voltage $V_A$\\
$V'_A = \frac{V_A}{L}$, early voltage process parameter $= 20V/\mu m$\\
$A_o = gm \cdot r_o$, Intrinsic gain for bjt and mosfet\\
$A_o = \frac{2V_A}{V_{ov}}$, Intrinsic gain\\
$A_o = \frac{2V'_AL}{V_{ov}}$, Intrinsic gain\\

\subsection{Proportionalities}
$I_D \propto V_{ov}^2$, drain current and overdrive voltage proportionality\\


%%%%%%%%%%%%%%%%%%_ Begin New Section _%%%%%%%%%%%%%%%%%%%
\section{\S1.2 - The PN Junction}




%%%%%%%%%%%%%%%%%%_ Begin New Section _%%%%%%%%%%%%%%%%%%%
\subsection{Notes}

\textbf{Common Source:} Input connected to gate, output connected to drain.\\
\textbf{Common Drain (Source Follower):} Input connected to gate, output connected to source.\\
\textbf{Common Gate:} Input connected to source, output connected to drain.\\

When $N_A >> N_D$, the semiconductor is p-type.\\
When $N_D >> N_A$, the semiconductor is n-type.\\

%%%%%%%%%%%%%%%%%%_ Begin New Section _%%%%%%%%%%%%%%%%%%%
\section{BJT Amplifier}

$g_m = 2\sqrt{K_n I_{DQ}}$,\\
$g_m = \frac{I_D}{V_{GS}}$,\\
$g_m = 2K_n (V_{GS}-V_{TH})$,\\
$g_m = \frac{I_C}{V_TH}$,\\
$g_m = \frac{I_C}{V_T}$\\
$r_o = \frac{1}{\lambda I_{DQ}}$,\\
$r_o = \frac{V_A}{I_C}$,\\
$r_\pi = \frac{V_T}{I_B}$,\\
$r_\pi = \frac{\beta}{g_m}$,\\
$A_v = -g_m \cdot R_C || R_L$, Voltage Gain Formula\\

%%%%%%%%%%%%%%%%%%_ Begin New Section _%%%%%%%%%%%%%%%%%%%
\subsection{Transistor DC Equivalent}
$V_{th} = \frac{V_{cc}}{R_1 + R_2} \cdot R_2$,\\
$R_{th} = R_1 || R_2$,\\
$V_{ce}(sat) \approxeq 0.2(typ)$,\\
$I_E \approxeq I_C$, In active region\\
$-\frac{1}{R_E-R_C}$, load line slope, where $R_C$ \& $R_E$ are from the AC or DC equivalent circuit. A load line plot is $I_C$ vs $V_{CE}$\\
$I_{RE} = I_B (\beta + 1) R_E$,\\

\subsection{Transistor formulas}
$I_C = \beta \cdot I_B$, Conduction Parameter\\
$I_C = I_S e^{\frac{V_{BE}}{V_T}}$,\\
$I_B = \frac{I_E}{\beta + 1}$,\\
$\alpha = \frac{I_C}{I_E}$, Current Ratio\\
$I_C = I_E - I_B$, Kirchhoff's Current Law\\
$V_{CE} = V_{BE} + V_{CB}$, Voltage Relationships\\
$I_C = I_{C0} \left( e^{\frac{V_{BE}}{V_T}} - 1 \right)$, BJT Current Equation\\
$I = I_0 \cdot \left( e^{\frac{V}{n \cdot V_T}} - 1 \right)$, Schottky Diode Equation\\
$I_D = \frac{1}{2} \mu_n C_{ox} \frac{W}{L} \left( V_{GS} - V_{TH} \right)^2$, MOSFET Drain Current Equation\\
$I_D = \mu_n C_{ox} \frac{W}{L} \left[ (V_{GS} - V_{TH})V_{DS} - \frac{V_{DS}^2}{2} \right]$, MOSFET Drain Current Equation (Triode Region)\\
$g_m = \sqrt{2 \mu_n C_{ox} \frac{W}{L} I_D}$, Transconductance Parameter\\
$A_v = -g_m \cdot R_D$, Voltage Gain Formula\\

%%%%%%%%%%%%%%%%%%_ Begin New Section _%%%%%%%%%%%%%%%%%%%
\section{MOS Field-Effect Transistor}


\subsection{N-Channel}
$v_{DS}(sat) = v_{GS} - V_{TN}$, Saturation Voltage, where $V_{TN}$ is the threshold voltage.\\
$i_D = K_n \left[2(v_{GS} - V_{TN} ) v_{DS} - v_{DS}^2\right]$, I-V Characteristic in non-saturation.\\
$i_D = K_n (v_{GS} - V_{TN})^2$, I-V Characteristic in saturation.\\
$C_{ox} = \epsilon_{ox} / t_{ox}$, Oxide capacitance per unit area.\\
$\epsilon_{ox} = (3.9)(\SI{8.85e-14}{F/cm})$, Oxide permittivity for Si devices.\\
$K_n = \frac{W \mu_n C_{ox}}{2L}$, Conduction Parameter\\
$K_n = \frac{k'_n}{2} \cdot \frac{W}{L}$, Conduction Parameter\\
$k'_n = \mu_nC_{ox}$, Process conduction parameter.\\
$\mu_n$, Electron mobility in the inversion layer.\\

\subsection{P-Channel}
$i_D = K_p \left[2(v_{SG} - V_{TP} ) v_{SD} - v_{SD}^2\right]$, I-V Characteristic in non-saturation.\\
$i_D = K_p (v_{SG} - V_{TP})^2$, I-V Characteristic in saturation.\\
$K_p = \frac{W \mu_p C_{ox}}{2L}$, Conduction Parameter\\
$K_p = \frac{k'_p}{2} \cdot \frac{W}{L}$, Conduction Parameter\\
$k'_p = \mu_p C_{ox}$\\


\section{EE General Formulae}
$rms = \frac{1}{\sqrt{2}}$,\\
$V = I \cdot R$, Ohm's law.\\
$P = V \cdot I$, DC Power.\\
$P = V \cdot I \cdot \cos(\theta)$, AC power.\\
$E = P \cdot t$, Energy.\\
$C = \frac{Q}{V}$, Capacitance.\\

$V = L \cdot \frac{di}{dt}$, Inductance.\\
$\tau = R \cdot C$, Time constant to reach 63.2\% of capacitors final voltage.\\
$\tau = \frac{L}{R}$, Time constant to reach 63.2\% of inductors final value.
$\frac{N_1}{N_2} = \frac{V_1}{V_2}$, Transformer turns ratio.\\
$V_{\text{peak}} = \sqrt{2} \cdot V_{\text{rms}}$, Peak AC Voltage.\\
$V_{\text{rms}} = \frac{V_{\text{peak}}}{\sqrt{2}}$, RMS AC Voltage.\\
$V_{\text{avg}} = \frac{1}{T} \int_0^T V(t) \, dt$, RMS AC Voltage.\\
$V_{\text{out}} = V_{\text{in}} \cdot \frac{R_2}{R_1 + R_2}$, voltage divider.\\
$R_{\text{eq}} = R_1 + R_2 + \ldots + R_n$, series resistors.\\
$\frac{1}{R_{\text{eq}}} = \frac{1}{R_1} + \frac{1}{R_2} + \ldots + \frac{1}{R_n}$, Parallel resistors.\\
$\frac{1}{C_{\text{eq}}} = \frac{1}{C_1} + \frac{1}{C_2} + \ldots + \frac{1}{C_n}$, Series capacitors.\\
$C_{\text{eq}} = C_1 + C_2 + \ldots + C_n$, parallel capacitors.\\


%%%%%%%%%%%%%%%%%%_ Begin New Section _%%%%%%%%%%%%%%%%%%%
\section{Convert Polar to Rectangular}

$$x = r \cos \theta$$
$$y = r \sin \theta$$

\section{Exact Slope of a Tangent Line}

$$\frac{dy}{dx} = \frac{dy/dt}{dx/dt}$$





%%%%%%%%%%%%%%%%%%_ Begin New Section _%%%%%%%%%%%%%%%%%%%
\section{Basic integration Rules}

$\int kf(u) du = k \int f(u) du + C$, $\int [f(u) \pm g(u)]du = \int f(u) du \pm \int g(u) du$, 
$\int du = u +C$, $\int u^n du = \frac{u^{n+1}}{n+1} +C, n \neq -1$, $\int \frac{du}{u} = \ln |u| +C$, $\int \frac{u}{du} = \frac{u^2}{2} +C$,
$\int e^u du = e^u +C$,$\int e^{4u} = \frac{e^{4u}}{4} +C$, $\int a^u du = \left( \frac{1}{\ln a} \right) a^u +C$, \\

\subsection{Some Integrals}

$\int \sin u du = -\cos u+C$, $\int \cos u du = \sin u +C$,
$\int \tan u du = -\ln| \cos u | +C$, $\int \cot u du = \ln| \sin u | +C$, $\int \sec u du = \ln| \sec u + \tan u | +C$,
$\int \csc u du = -\ln| \csc u + \cot u | +C$, $\int \sec^2 u du = \tan u +C$, $\int \csc^2 u du = -\cot u +C$,
$\int \sec u \tan u du = \sec u +C$, $\int \csc u \cot u du = -\csc u +C$, $\int \frac{du}{\sqrt{a^2-u^2}} = \arcsin \frac{u}{a} +C$,
$\int \frac{du}{a^2+u^2} = \frac{1}{a} \arctan \frac{u}{a} +C$,
$\int \frac{du}{u\sqrt{u^2-a^2}} = \frac{1}{a} $arcsec$ \frac{|u|}{a} +C$, $\int \sin 3x = -\frac{1}{3} \cos 3x$, $\int e^{-4x} = \frac{e^{-4x}}{-4}$

$\int k dx = kx+C$, $\int x dx = \frac{1}{2}x^2 +C$, $\int x^2 dx = \frac{1}{3} x^3 +C$, $\int \frac{1}{x} dx = \ln|x| + C$,
$\int e^x dx = e^x +C$,$\int k^u du = \frac{k^u}{\ln u} +C$,  $\int \ln x dx = x \ln x - x +C$, $\int \cos x dx = \sin x +C$,
$\int \sin x dx = -\cos x +C$, $\int \sec^2 x dx = \tan x +C$, $\int x^n dx = \frac{x^{n+1}}{n+1} +C$, $\int \tan x = - \ln( \cos x ) +C$,


%%%%%%%%%%%%%%%%%%_ Begin New Section _%%%%%%%%%%%%%%%%%%%
\subsection{Integration by Parts}

$$\int u dv = uv - \int v du$$




%%%%%%%%%%%%%%%%%%_ Begin New Section _%%%%%%%%%%%%%%%%%%%
\section{Some Identities}

$\sin2x = 2 \sin x \cos x$

\subsection{Pythagorean:}
$\sin^2 x+ \cos^2 x = 1$, $1 + \tan^2 x = \sec^2x$, $1 + cot^2 x = \csc^2 x$ 

\subsection{Reciprocal:}
$\sin x = \frac{1}{\csc x}$, $\cos x = \frac{1}{\sec x}$, $\tan x = \frac{\sin x}{\cos x} = \frac{1}{\cot x}$\\
$\csc x = \frac{1}{\sin x}$, $\sec x = \frac{1}{\cos x}$, $\cot x = \frac{\cos x}{\sin x} = \frac{1}{\tan x}$ 

\subsection{Half Angle:}
$\sin^2 x = \frac{1}{2} (1 - \cos 2x)$, $\cos^2 x = \frac{1}{2} ( 1 + \cos 2x)$

%%%%%%%%%%%%%%%%%%_ Begin New Section _%%%%%%%%%%%%%%%%%%%
\section{Additional Notes:}


$\ln ( x * y ) = \ln(x) + \ln(y)$, $\ln ( x / y ) = \ln( x ) - \ln ( y )$

$\ln x^a = a \ln x$, $\tan \theta = \frac{\sin \theta}{\cos \theta}$\\

$ax^2+bx+c = 0$, $x= \frac{-b \pm \sqrt{b^2-4ac}}{2a}$\\

$\ln a = c \equiv e^c = a$\\

$\sqrt[n]{a} = a^\frac{1}{n}$, $a^{-n} = \frac{1}{a^n}$, $\sqrt[n]{a^m} = a^\frac{m}{n}$, $a^0 = 1$, $\left( a^m \right)^n = a^{mn}$, 
$a^m * a^n = a^{m + n}$, $\frac{ a^m }{ a^n } = a^{m-n}$, Rewrite $\sqrt{5x}$ as $\sqrt{5} \sqrt{x}$, \\





\section{Some Derivatives:}

$\frac{d}{du} \sin u = (\cos u)u'$, $\frac{d}{du} \cos u = -(\sin u)u'$, $\frac{d}{du} \tan u = (\sec^2 u)u'$,
$\frac{d}{du} \cot u = -(\csc^2 u)u'$, $\frac{d}{du} \sec u = (\sec u \tan u)u'$, $\frac{d}{du} \csc u = -(\csc u \cot u)u'$,
$\frac{d}{du} \arcsin u = \frac{u'}{\sqrt{1 - u^2}}$, $\frac{d}{du} \arccos u = \frac{-u'}{\sqrt{1 - u^2}}$,
$\frac{d}{du} \arctan u = \frac{u'}{1 + u^2}$, $\frac{d}{du} $arccot $ u = \frac{-u'}{1 + u^2}$, \\
$\frac{d}{du} $arcsec $ u = \frac{u'}{|u|\sqrt{u^2 - 1}}$,
$\frac{d}{du} $arccsc $ u = \frac{-u'}{|u|\sqrt{u^2 - 1}}$

$\frac{d}{du}[\ln{u}] = \frac{1}{u}u'$, $\frac{d}{dx}[e^{-x}] = -e^{-x}$, $e^{\ln a} = a$ \\
$\frac{d}{du}[\sqrt{u}] = \frac{u'}{2 \sqrt{u}}$, $e^{3x} = 3e^{3x}$, $\frac{d}{dx}\left[ x \right] = 1$, $\frac{d}{dx}\left[ c \right] = 0$, 
$\frac{d}{du}[ \frac{1}{u} ] = \frac{1}{u^2}$, $\frac{du}{u} = \ln |u|$,


\includegraphics[width = \linewidth]{./img/unitCircle}






\end{multicols}
\end{document}
