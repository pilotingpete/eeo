\documentclass{article}
%\usepackage{blindtext}
\usepackage[letterpaper, total={6.5in, 9in}]{geometry}
%\usepackage{tabularray}
%\usepackage{hyperref}
%\usepackage{xcolor}
\usepackage{amsmath}
\usepackage{siunitx}
\usepackage{graphicx}
\usepackage{pdfpages}
\usepackage{float}
\usepackage[normalem]{ulem}
\usepackage{booktabs}

\graphicspath{ {./img/} }
\newcommand \imgWidthFactor{0.8}


\begin{document}
	
\begin{titlepage}
	\centering
	\includegraphics[width=0.45\textwidth]{sbu}\par\vspace{1cm}
	{\LARGE \textsc{EEO311}\par}
	\vspace{1cm}
	{\Large \textsc{Spring 2024}\par}
	\vspace{1.5cm}
	{\huge\bfseries Analysis of the common-source differential stage\par}
	\vspace{2cm}
	{\Large\itshape Pete Mills\\ID: 115009163\par}
	\vfill
	Associate Professor\par
	Dmitri \textsc{Donetski}

	\vfill

% Bottom of the page
	{\large \today\par}
\end{titlepage}

% Include the original assignment
	\newcommand{\assName}{EEO311 Simulation assignment 2_AnalysisOfCommonSourceGainStage}

	\includepdf[pages=-,pagecommand=\section*{Copy of Original Assignment}]{\assName}
	%\includepdf[pages=2-,pagecommand={}]{\assName}
	
	%\listoffigures
	
% Begin report

	\section*{Overview}
	
	 In simulation 2 our primary objective is to observe how changes in mosfet construction parameters can influence a design. Here we made adjustments to mosfets used in the implementation of a differential amplifier. By making adjustments to the width parameter and current mirror reference resistor we are able to tune the amplifier to improve efficiency and minimize distortion.
	
	\section{Assignment}

	\subsection{Part A}

		\begin{figure}[H]
			\centering
			\includegraphics[width=\imgWidthFactor\textwidth]{a1}
			\caption{Bias point simulation with reference current resistor value set to \SI{663}{\kohm}.}
		\end{figure}

	\subsection{Part B}

		\begin{figure}[H]
			\centering
			\includegraphics[width=\imgWidthFactor\textwidth]{b1}
			\caption{M10 width changed to \SI{51}{\um}.}
		\end{figure}

		\begin{figure}[H]
			\centering
			\includegraphics[width=\imgWidthFactor\textwidth]{b2}
			\caption{DC Transfer characteristic for Vout1 reveals a range $\approx \SIrange{-90}{100}{\mV}$.}
		\end{figure}

	\subsection{Part C}

		\begin{figure}[H]
			\centering
			\includegraphics[width=\imgWidthFactor\textwidth]{c1}
			\caption{Cutoff frequency: \SI{114.3}{\MHz}. Gain: \SI{24.3}{}. GBP: \SI{1.87}{\GHz}.}
		\end{figure}

	\subsection{Part D}

		\begin{figure}[H]
			\centering
			\includegraphics[width=\imgWidthFactor\textwidth]{d1}
			\caption{Cutoff frequency: \SI{17.3}{\MHz}. Gain: \SI{24.3}{}. GBP: \SI{282.9}{\GHz}. Unity Gain: \SI{286.5}{\MHz}}
		\end{figure}
	
	\subsection{Part E}

		\begin{figure}[H]
			\centering
			\includegraphics[width=\imgWidthFactor\textwidth]{e1}
			\caption{DC transfer characteristic.}
		\end{figure}

		\begin{figure}[H]
			\centering
			\includegraphics[width=\imgWidthFactor\textwidth]{e2}
			\caption{Schematic}
		\end{figure}

		\begin{figure}[H]
			\centering
			\includegraphics[width=\imgWidthFactor\textwidth]{e3}
			\caption{Cutoff frequency: \SI{23.3}{\MHz}. Gain: \SI{24.3}{}. GBP: \SI{381.4}{\GHz}. Unity Gain: \SI{386.5}{\MHz}}
		\end{figure}

	\subsection{Part F}

		\begin{figure}[H]
			\centering
			\includegraphics[width=\imgWidthFactor\textwidth]{f1}
			\caption{Cutoff frequency: \SI{2.5}{\MHz}. Unity Gain: \SI{40.4}{\MHz}, Pole 1: \SI{2.5}{\MHz}, Pole 2: \SI{200}{\MHz} }
		\end{figure}

	\subsection{Part G}
	
	\begin{table}[htbp]
		\centering
		\caption{Simulation Data Summary}
		\label{tab:state_machine}
		\begin{tabular}{lllllll}
			\toprule
			\textbf{Case} & \textbf{Bias Current} & \textbf{Gain (SE)} & \textbf{Gain (DO)} & \textbf{Bandwidth} & \textbf{GBP} & \textbf{Unity Gain}\\
			\midrule
			3D & \SI{48.1}{\uA} & \SI{24.3}{} & \SI{32.5}{} & \SI{17.3}{\MHz} & \SI{282.9}{\GHz} & \SI{286.5}{\MHz}\\
			3E & \SI{48.1}{\uA} & \SI{24.3}{} & \SI{32.7}{} & \SI{23.3}{\MHz} & \SI{381.4}{\GHz} & \SI{386.5}{\MHz}\\
			3F & \SI{71.3}{\uA} & \SI{24.3}{} & \SI{32.7}{} & \SI{2.5}{\MHz}  & \SI{41.3}{\MHz} & \SI{40.4}{\MHz}\\
			\bottomrule
		\end{tabular}
	\end{table}

	\section{Bonus}

	I will return to complete if time permits.


\end{document}
