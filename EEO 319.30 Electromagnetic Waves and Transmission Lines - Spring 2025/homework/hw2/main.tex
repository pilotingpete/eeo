\documentclass{article}
%\usepackage{blindtext}
\usepackage[letterpaper, total={6.5in, 9in}]{geometry}
%\usepackage{tabularray}
%\usepackage{hyperref}
%\usepackage{xcolor}
\usepackage{amsmath}
\usepackage{siunitx}
\usepackage{graphicx}
\graphicspath{ {./img/} }
\usepackage{pdfpages}
\usepackage{float}
\usepackage[normalem]{ulem}


\begin{document}
	
\begin{titlepage}
	\centering
	\includegraphics[width=0.45\textwidth]{sbu}\par\vspace{1cm}
	{\LARGE \textsc{EEO319}\par}
	\vspace{1cm}
	{\Large \textsc{Spring 2025}\par}
	\vspace{1.5cm}
	{\huge\bfseries Electromagnetic Waves \& Transmission Lines\par}
	\vspace{2cm}
	{\Large\itshape Pete Mills\\ID: 115009163\par}
	\vfill
	Professor\par
	Jayant \textsc{Parekh}

	\vfill

% Bottom of the page
	{\large \today\par}
\end{titlepage}

% Include the original assignment
	\newcommand{\assName}{020925 Homework 2 (1).pdf}

	\includepdf[pages=-,pagecommand=\section*{Copy of Original Assignment}]{\assName}
	%\includepdf[pages=2-,pagecommand={}]{\assName}
	
	%\listoffigures
	
% Begin report

\section{Wave Analysis}

\subsection{Problem 1}
A uniform plane wavefunction has the instantaneous expression:
\begin{equation}
    \Phi(x,t) = 5 \sin(200\pi t + 0.4\pi x + 300)
\end{equation}
Identify or calculate the following:

\begin{enumerate}
    \item The radian frequency $\omega$ of the wave in rad/s:
    \begin{equation}
        \omega = 200\pi \text{ rad/s}
    \end{equation}
    \item The frequency $f$ in Hz of the wave:
    \begin{equation}
        f = \frac{\omega}{2\pi} = \frac{200\pi}{2\pi} = 100 \text{ Hz}
    \end{equation}
    \item The direction of propagation of the wave:
    \newline The wave propagates in the negative $x$-direction since the wave phase term is $(\omega t + kx + \phi)$.
    \item The wavelength $\lambda$ in meters:
    \begin{equation}
        k = 0.4\pi, \quad \lambda = \frac{2\pi}{k} = \frac{2\pi}{0.4\pi} = 5 \text{ m}
    \end{equation}
    \item The amplitude $A$ of the wave:
    \begin{equation}
        A = 5
    \end{equation}
    \item The phase velocity $v_{ph}$ in m/s of the wave:
    \begin{equation}
        v_{ph} = \frac{\omega}{k} = \frac{200\pi}{0.4\pi} = 500 \text{ m/s}
    \end{equation}
\end{enumerate}

\subsection{Problem 2}
A uniform plane scalar wave represented by the function $\Phi(x,t)$ is given with the following properties:
\begin{itemize}
    \item The wave propagates in the $-x$ direction.
    \item Wave amplitude = 10.
    \item Wave frequency = 500 Hz.
    \item Wave’s phase velocity = 100 m/s.
    \item The wavefunction $\Phi(0,0)$ at $x = 0$ and $t = 0$ has the value $\Phi(0,0) = 5$.
\end{itemize}
Find the expression $\Phi(x,t)$ of the wave function.

\begin{equation}
    \omega = 2\pi f = 2\pi \times 500 = 1000\pi \text{ rad/s}
\end{equation}
\begin{equation}
    k = \frac{\omega}{v_{ph}} = \frac{1000\pi}{100} = 10\pi \text{ rad/m}
\end{equation}
Since the wave propagates in the $-x$ direction, the wavefunction takes the form:
\begin{equation}
    \Phi(x,t) = 10 \sin(\omega t + kx + \phi)
\end{equation}
To satisfy $\Phi(0,0) = 5$:
\begin{equation}
    10 \sin(\phi) = 5 \Rightarrow \sin(\phi) = 0.5 \Rightarrow \phi = \frac{\pi}{6}
\end{equation}
Thus, the wave function is:
\begin{equation}
    \Phi(x,t) = 10 \sin(1000\pi t + 10\pi x + \frac{\pi}{6})
\end{equation}

\subsection{Problem 3}
Consider two scalar UPW’s of the same amplitude which have incrementally different frequencies and thus incrementally different wavelengths. Both waves are given to be propagating in the $+z$ direction.

Let the two waves be:
\begin{equation}
    \Phi_1 = A \cos(\omega_1 t - k_1 z)
\end{equation}
\begin{equation}
    \Phi_2 = A \cos(\omega_2 t - k_2 z)
\end{equation}
Using trigonometric identities, their sum can be rewritten as an amplitude-modulated wave:
\begin{equation}
    \Phi = \Phi_1 + \Phi_2 = 2A \cos \left( \frac{\Delta \omega}{2} t - \frac{\Delta k}{2} z \right) \cos \left( \omega_c t - k_c z \right)
\end{equation}
where:
\begin{equation}
    \omega_c = \frac{\omega_1 + \omega_2}{2}, \quad k_c = \frac{k_1 + k_2}{2},
\end{equation}
\begin{equation}
    \Delta \omega = \omega_2 - \omega_1, \quad \Delta k = k_2 - k_1
\end{equation}
The carrier wave propagates with phase velocity:
\begin{equation}
    v_p = \frac{\omega_c}{k_c}
\end{equation}
The envelope wave propagates with group velocity:
\begin{equation}
    v_g = \frac{d\omega}{dk}
\end{equation}
This shows that the interference of two UPWs results in an amplitude-modulated signal with the carrier wave moving at the phase velocity and the envelope moving at the group velocity.


\end{document}
