\documentclass{include/src/sbu-report}  % Use the custom class
% ===========================
% Pete's Macros
% ===========================

\usepackage{amsmath}
\usepackage{siunitx}
\usepackage{cancel}

% === UNIT CONVERSION AND CANCELLATION ===

% Define a macro for unit cancellation
\newcommand{\unitcancel}[2]{\frac{#1 \cancel{#2}}{#2}}

% Define a macro for unit conversion
\newcommand{\convert}[3]{\frac{#1 \cancel{#2}}{#3}}

% Define a macro for general equations (to make writing consistent)
\newcommand{\eqn}[1]{\begin{align} #1 \end{align}}

% === INCLUDING IMAGES ===

% === Image Insertion Macro ===
% Usage: \img{filename}{caption}{label}
% 
% #1 = Image file name
% #2 = Caption text for the figure
% #3 = Label for referencing (used with \ref{})
\newcommand{\img}[3]{
\begin{figure}[H]
    \centering
    \includegraphics[width=0.65\textwidth]{#1}
    \caption{#2}
    \label{fig:#3}
\end{figure}
}


% === CIRCUIT ANALYSIS ===

\newcommand{\ohmslaw}{V = IR}
\newcommand{\power}{P = VI}
\newcommand{\powerR}{P = \frac{V^2}{R}}
\newcommand{\powerI}{P = I^2 R}
\newcommand{\impedance}{Z = R + jX}
\newcommand{\admittance}{Y = \frac{1}{Z}}
\newcommand{\xc}{X_C = \frac{1}{\omega C}}
\newcommand{\xl}{X_L = \omega L}
\newcommand{\voltagedivider}{V_{out} = V_{in} \times \frac{R_2}{R_1 + R_2}}
\newcommand{\currentdivider}{I_{x} = I_{total} \times \frac{R_{total}}{R_x}}
\newcommand{\kcl}{\sum I = 0}
\newcommand{\kvl}{\sum V = 0}
\newcommand{\vrms}{V_{rms} = \frac{V_{peak}}{\sqrt{2}}}
\newcommand{\omegaeq}{\omega = 2\pi f}

% === BATTERY MODELING AND MANAGEMENT ===

\newcommand{\batteryECM}{V_{cell} = OCV - R_{int} I - V_{RC}}
\newcommand{\batteryThermal}{Q = mC_p \frac{dT}{dt} + hA(T - T_{env})}
\newcommand{\batterySOC}{SOC = SOC_0 - \int_0^t \frac{I}{C} dt}
\newcommand{\batterySOH}{SOH = \frac{C_{actual}}{C_{nominal}} \times 100\%}
\newcommand{\batteryPowerLimit}{P_{max} = \frac{V^2}{R}}
\newcommand{\batteryAging}{C_{loss} = C_0 \left(1 - \frac{t}{t_{end}}\right)}
\newcommand{\cellBalancing}{I_{bal} = \frac{V_{cell} - V_{target}}{R_{bal}}}

% === POWER ELECTRONICS ===

\newcommand{\dcdcGain}{V_{out} = V_{in} \times D}
\newcommand{\buckVoltage}{V_{out} = D V_{in}}
\newcommand{\boostVoltage}{V_{out} = \frac{V_{in}}{1 - D}}
\newcommand{\bidirectionalDCDC}{P_{out} = P_{in} - P_{loss}}
\newcommand{\powerFactorCorrection}{PF = \frac{P_{real}}{S}}
\newcommand{\resonantFrequency}{f_r = \frac{1}{2\pi\sqrt{LC}}}

% === MOTOR DRIVES ===

\newcommand{\inductionTorque}{T = \frac{3}{2} P \frac{\phi I_r}{\omega_s}}
\newcommand{\pmsmTorque}{T = \frac{3}{2} P \frac{\phi I_q}{\omega_s}}
\newcommand{\synchronousSpeed}{N_s = \frac{120f}{P}}
\newcommand{\rotorInductance}{L_r = L_m + L_{lr}}
\newcommand{\dqTransform}{
\begin{bmatrix} d \\ q \end{bmatrix} =
\begin{bmatrix} \cos\theta & \sin\theta \\ -\sin\theta & \cos\theta \end{bmatrix}
\begin{bmatrix} \alpha \\ \beta \end{bmatrix}
}

% === CONTROL SYSTEMS ===

\newcommand{\pid}{C(s) = K_p + K_i \frac{1}{s} + K_d s}
\newcommand{\rootLocus}{\text{Plot of } 1 + G(s)H(s) = 0}
\newcommand{\bodeGainMargin}{G_m = -1/\text{Gain at } \omega_c}
\newcommand{\nyquist}{N = P - Z}
\newcommand{\kalmanUpdate}{\hat{x}_k = \hat{x}_{k|k-1} + K_k (z_k - H\hat{x}_{k|k-1})}
\newcommand{\statespace}{
    \begin{cases}
        \dot{x}(t) = Ax(t) + Bu(t) \\
        y(t) = Cx(t) + Du(t)
    \end{cases}
}
\newcommand{\mpc}{\min \sum_{k=0}^{N} (x_k^T Q x_k + u_k^T R u_k)}

% === SYSTEM IDENTIFICATION ===

\newcommand{\transferFunction}{H(s) = \frac{Y(s)}{U(s)}}
\newcommand{\leastSquares}{\theta = (X^T X)^{-1} X^T Y}
\newcommand{\stateEstimation}{\hat{x}_{k+1} = A \hat{x}_k + B u_k + K (y_k - C \hat{x}_k)}

% === OPTIMIZATION AND ROBUST CONTROL ===

\newcommand{\lqr}{J = \int_0^\infty (x^T Q x + u^T R u) dt}
\newcommand{\hInfinity}{\min \max \|T_{zw}(s)\|_\infty}
\newcommand{\convexOptimization}{\min f(x) \quad \text{subject to } g_i(x) \leq 0}

% === EMBEDDED SYSTEMS ===

\newcommand{\samplingTime}{T_s = \frac{1}{f_s}}
\newcommand{\adcResolution}{V_{LSB} = \frac{V_{ref}}{2^N}}
\newcommand{\pwmDutyCycle}{D = \frac{t_{on}}{t_{on} + t_{off}}}

% === POWER SYSTEMS ===

\newcommand{\powerBalance}{P_{in} = P_{out} + P_{loss}}
\newcommand{\mppt}{\frac{dP}{dV} = 0}
\newcommand{\gridInverter}{P = V I \cos \theta}

% === PHASORS AND COMPLEX POWER ===

\newcommand{\phasor}[2]{#1 \angle #2^\circ}
\newcommand{\complexpower}{S = P + jQ}
\newcommand{\powerfactor}{PF = \cos \theta}

% Load macros only once
\providecommand{\loadedmacros}{}
\renewcommand{\loadedmacros}{loaded}

% ===========================
% End of Macros
% ===========================
          % Import macros

% Set Title Page Parameters
\setcoursename{EEO399}
\setsemester{Spring 2025}
\setreporttitle{Electromagnetic Waves \& Transmission Lines}
\setprofessor{Jayant Parekh}
\setstudentname{Pete Mills}
\setstudentid{115009163}
\setreportdate{\today}

\begin{document}

% Optional Includes
\includeassignment{021325 Homework 3.pdf}
\includelogo{sbu}

\maketitle  % Generate the title page

% Begin Report

\section{Problem 1: Uniform Plane Electromagnetic Wave}
Given the parameters:
\begin{itemize}
    \item Frequency: $f = 6$ GHz
    \item Relative Permittivity: $\varepsilon_r = 9$
    \item Linear Polarization: $y$ direction
    \item Amplitude of Electric Field: $10$ V/m
    \item Initial Electric Field: $E(0,0) = 1$ V/m
\end{itemize}

The wave number is given by:
\begin{equation}
    k = \frac{\omega}{v_p} = \frac{2\pi f}{c / \sqrt{\varepsilon_r}} = \frac{2\pi (6 \times 10^9)}{3 \times 10^8 / 3} = 4\pi \times 10^2 \text{ rad/m}
\end{equation}

The phase velocity is:
\begin{equation}
    v_p = \frac{c}{\sqrt{\varepsilon_r}} = \frac{3 \times 10^8}{3} = 10^8 \text{ m/s}
\end{equation}

\subsection{Instantaneous Electric Field Expression}
\begin{equation}
    E(x,t) = 10 \cos(\omega t - kx + \phi) \hat{y}
\end{equation}

\subsection{Phasor Electric Field Expression}
\begin{equation}
    \tilde{E}(x) = 10 e^{-jkx} \hat{y}
\end{equation}

\subsection{Instantaneous Magnetic Field Expression}
The intrinsic impedance of the medium is:
\begin{equation}
    \eta = \frac{\eta_0}{\sqrt{\varepsilon_r}} = \frac{377}{3} \approx 125.67 \text{ } \Omega
\end{equation}

\begin{equation}
    H(x,t) = \frac{E_0}{\eta} \cos(\omega t - kx + \phi) \hat{z}
\end{equation}

\subsection{Phasor Magnetic Field Expression}
\begin{equation}
    \tilde{H}(x) = \frac{10}{125.67} e^{-jkx} \hat{z}
\end{equation}

\section{Problem 2: Plane Wave in a Lossless Dielectric Medium}
Given the electric field:
\begin{equation}
    E(r,t) = (\hat{y} + \hat{z} \sqrt{3}) \cos(24\pi \times 10^9 t + 160\sqrt{3}\pi y - 160\pi z + 200 )
\end{equation}

\subsection{Wavelength Calculation}
The wave number magnitude is:
\begin{equation}
    k = \sqrt{(160\sqrt{3}\pi)^2 + (160\pi)^2}
\end{equation}
\begin{equation}
    k = 320\pi \text{ rad/m}
\end{equation}
The wavelength is:
\begin{equation}
    \lambda = \frac{2\pi}{k} = \frac{2\pi}{320\pi} = \frac{1}{160} \text{ m} \approx 6.25 \text{ mm}
\end{equation}

\subsection{Propagation Direction}
\begin{equation}
    \hat{k} = \frac{(160\sqrt{3}\pi \hat{y} - 160\pi \hat{z})}{320\pi} = \frac{1}{2} \hat{y} - \frac{1}{2} \hat{z}
\end{equation}
Thus, the wave propagates in the direction $\frac{1}{2} \hat{y} - \frac{1}{2} \hat{z}$.

\subsection{Relative Refractive Index}
\begin{equation}
    n_r = \frac{c}{v_p} = \frac{3 \times 10^8}{v_p}
\end{equation}

\subsection{Instantaneous Magnetic Field Expression}
\begin{equation}
    H(r,t) = \frac{1}{\eta} \left( \hat{k} \times E(r,t) \right)
\end{equation}

\subsection{Phasor Expression for the Electric Field}
\begin{equation}
    \tilde{E}(r) = (\hat{y} + \hat{z} \sqrt{3}) e^{j(160\sqrt{3}\pi y - 160\pi z + 200)}
\end{equation}

\section{Problem 3: Electromagnetic Wave in Vacuum}
Given the phasor expression:
\begin{equation}
    E(y) = e^{-jky} \hat{x} - j 10 e^{-jky} \hat{z}
\end{equation}

\subsection{Instantaneous Electric Field Expression}
\begin{equation}
    E(y,t) = \operatorname{Re} \{ (e^{-jky} \hat{x} - j 10 e^{-jky} \hat{z}) e^{j\omega t} \}
\end{equation}

\subsection{Instantaneous Magnetic Field Expression}
\begin{equation}
    H(y,t) = \frac{1}{\eta} \left( \hat{k} \times E(y,t) \right)
\end{equation}

\subsection{Polarization of the Wave}
Elliptical polarization.

\subsection{Phasor Expression for the Magnetic Field}
\begin{equation}
    \tilde{H}(y) = \frac{1}{\eta} \left( \hat{k} \times \tilde{E}(y) \right)
\end{equation}

\end{document}
