\documentclass{article}
\usepackage{amsmath, amssymb, graphicx}
\usepackage{geometry}
\geometry{a4paper, margin=1in}

\title{Lesson 3: Uniform Plane Electromagnetic Waves (UPEMWs)}
\author{}
\date{}

\begin{document}

\maketitle

\section{Key Learning Objectives}
\begin{itemize}
    \item Understand UPEMWs propagating in a lossless infinite medium (vacuum or dielectric).
    \item Recognize that a UPEMW consists of two UPWs: one for the electric field and one for the magnetic field.
    \item Learn the fundamental properties of UPEMWs.
\end{itemize}

\section{Wave Function of UPEMWs}
UPEMWs are vector wave functions, meaning that both electric and magnetic fields have direction in space in addition to their spatial and temporal dependence. A simple case of a UPEMW propagating in the $z$-direction is expressed as:
\begin{align}
    E(z,t) &= i_E E_0 \cos(\omega t - kz) \quad \text{(1a)} \\
    H(z,t) &= i_H H_0 \cos(\omega t - kz) \quad \text{(1b)}
\end{align}
where:
\begin{itemize}
    \item $i_E$ and $i_H$ are unit vectors representing the directions of the electric and magnetic fields, respectively.
    \item $E_0$ and $H_0$ are amplitude constants.
    \item $k$ is the wavenumber.
    \item $\omega$ is the angular frequency.
\end{itemize}
A more general form of the wave function is:
\begin{align}
    E(z,t) &= i_E E_0 \cos(\omega t - \mathbf{k} \cdot \mathbf{r}) \quad \text{(2a)} \\
    H(z,t) &= i_H H_0 \cos(\omega t - \mathbf{k} \cdot \mathbf{r}) \quad \text{(2b)}
\end{align}
where $\mathbf{k}$ is the wave vector.

\section{Properties of UPEMWs}
\subsection{Property 1: Synchronization of Fields}
The space-time factor $\cos(\omega t - \mathbf{k} \cdot \mathbf{r})$ is the same for both $E$ and $H$, meaning:
\begin{itemize}
    \item Electric and magnetic fields are in phase.
    \item Their maxima and minima occur at the same points in space and time.
\end{itemize}

\subsection{Property 2: Right-Handed Triad of Unit Vectors}
The unit vectors $i_E$, $i_H$, and $i_K$ form a right-handed coordinate system:
\begin{align}
    i_E \times i_H &= i_K \quad \text{(3a)} \\
    i_H \times i_K &= i_E \quad \text{(3b)} \\
    i_K \times i_E &= i_H \quad \text{(3c)}
\end{align}
These relationships provide a mnemonic for determining an unknown unit vector given two known vectors. The right-hand rule can also be used to determine the direction of the unknown unit vector.

\subsection{Property 3: Relationship Between $E_0$ and $H_0$}
The amplitudes of the electric and magnetic fields are related by the characteristic impedance $\eta$ of the medium:
\begin{equation}
    \frac{E_0}{H_0} = \eta \quad \text{(4)}
\end{equation}
where $\eta$ is given by:
\begin{equation}
    \eta = \sqrt{\frac{\mu_0 \mu_r}{\epsilon_0 \epsilon_r}} \quad \text{(5)}
\end{equation}
with:
\begin{itemize}
    \item $\mu_0 = 4\pi \times 10^{-7} \text{ H/m}$ (permeability of free space).
    \item $\mu_r$ = relative permeability of the medium ($\mu_r = 1$ for a nonmagnetic medium).
    \item $\epsilon_0 = 8.854 \times 10^{-12} \text{ F/m}$ (permittivity of free space).
    \item $\epsilon_r$ = relative permittivity (dielectric constant of the medium).
\end{itemize}
For free space:
\begin{equation}
    \eta_0 = 377 \Omega \quad \text{(6)}
\end{equation}

\section{Phase Velocity of UPEMWs}
The phase velocity of UPEMWs is given by:
\begin{equation}
    v_p = \frac{c}{n_r} \quad \text{(7)}
\end{equation}
where $n_r$ is the relative refractive index, defined as the velocity reduction factor relative to the speed of light in vacuum ($c = 3 \times 10^8$ m/s). The refractive index is:
\begin{equation}
    n_r = \sqrt{\mu_r \epsilon_r} \quad \text{(8)}
\end{equation}

\section{Conclusion}
\begin{itemize}
    \item A UPEMW consists of an electric and magnetic field propagating together.
    \item Both fields are in phase and perpendicular to each other and to the direction of propagation.
    \item The amplitudes of the fields are related through the characteristic impedance of the medium.
    \item The phase velocity depends on the refractive index of the medium.
\end{itemize}

\end{document}
