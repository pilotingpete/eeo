\documentclass{article}
\usepackage{amsmath, amssymb, graphicx}
\usepackage{geometry}
\geometry{a4paper, margin=1in}

\title{Lesson 4: Phasor Wavefunctions}
\author{}
\date{}

\begin{document}

\maketitle

\section{Key Learning Objectives}
\begin{itemize}
    \item Understand phasors as a tool for simplifying sinusoidal steady-state analysis.
    \item Learn how phasors simplify mathematical operations on sinusoidal signals.
    \item Apply phasor techniques to analyze electrical circuits.
\end{itemize}

\section{Phasors in Steady-State Sinusoidal Analysis}
Phasors allow us to represent sinusoidal signals using complex numbers, greatly simplifying calculations such as addition, subtraction, multiplication, and division.

Consider the series LR circuit in Fig. 1 driven by a sinusoidal voltage source. The circuit's governing equation, derived from Kirchhoff's Voltage Law (KVL), is:
\begin{equation}
    L \frac{di(t)}{dt} + R i(t) = V_{\text{m}} \cos(\omega t)
\end{equation}
where:
\begin{itemize}
    \item $R$ is resistance,
    \item $L$ is inductance,
    \item $V_{\text{m}}$ is the voltage amplitude,
    \item $\omega$ is the angular frequency.
\end{itemize}

\section{Trigonometric Solution}
The steady-state solution for $i(t)$ must have the same frequency as the forcing function:
\begin{equation}
    i(t) = I_m \cos(\omega t - \phi)
\end{equation}
Substituting this into the differential equation and solving for $I_m$ and $\phi$ gives:
\begin{align}
    I_m &= \frac{V_m}{\sqrt{R^2 + \omega^2 L^2}} \quad \text{(Magnitude)} \\
    \tan \phi &= \frac{\omega L}{R} \quad \text{(Phase angle)}
\end{align}
Thus, the final expression for $i(t)$ is:
\begin{equation}
    i(t) = \frac{V_m}{\sqrt{R^2 + \omega^2 L^2}} \cos(\omega t - \phi)
\end{equation}

\section{Phasor Solution}
Rewriting the governing equation in complex form:
\begin{equation}
    L \frac{d i_x(t)}{dt} + R i_x(t) = V_m e^{j \omega t}
\end{equation}
Solving this algebraically gives:
\begin{equation}
    I = \frac{V_m}{R + j \omega L}
\end{equation}
Expressing in polar form:
\begin{equation}
    I = \frac{V_m}{\sqrt{R^2 + \omega^2 L^2}} e^{-j \phi}
\end{equation}
Using Euler's formula:
\begin{equation}
    i(t) = \Re\{ I e^{j \omega t} \} = \frac{V_m}{\sqrt{R^2 + \omega^2 L^2}} \cos(\omega t - \phi)
\end{equation}
which matches the trigonometric result.

\section{Phasors and the Frequency Domain}
A sinusoidal signal can be represented either in the time domain or frequency domain:
\begin{itemize}
    \item Time domain: $i(t) = I_m \cos(\omega t - \phi)$
    \item Frequency domain: $I = \frac{V_m}{R + j \omega L}$
\end{itemize}
Moving to the frequency domain simplifies many calculations, making it a useful technique for circuit analysis.

\section{Conclusion}
\begin{itemize}
    \item Phasors convert differential equations into algebraic equations, simplifying steady-state sinusoidal analysis.
    \item The magnitude and phase of the solution are easily obtained in phasor form.
    \item Moving between the time and frequency domains enhances problem-solving efficiency.
\end{itemize}

\end{document}
