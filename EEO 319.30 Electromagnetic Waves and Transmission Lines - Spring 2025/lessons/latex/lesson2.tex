\documentclass{article}
\usepackage{amsmath, amssymb, graphicx}
\usepackage{geometry}
\geometry{a4paper, margin=1in}

\title{Lesson 2: UPW Propagating in an Arbitrary Direction in Space}
\author{}
\date{}

\begin{document}

\maketitle

\section{Key Learning Objectives}
\begin{itemize}
    \item Understand how to write and interpret the expression for a UPW propagating in an arbitrary direction.
    \item Extend the understanding from Lesson 1, which covered waves along Cartesian coordinate axes, to waves in any direction.
\end{itemize}

\section{UPW Wave Function for Propagation in an Arbitrary Direction}
For simplicity, consider a uniform plane wave (UPW) propagating in an arbitrary direction in the xz-plane. The wave function is given by:
\begin{equation}
    \Phi(x,z,t) = A \cos(\omega t - \mathbf{k} \cdot \mathbf{r})
\end{equation}
where:
\begin{itemize}
    \item $\mathbf{r} = x \hat{i}_x + z \hat{i}_z$ is the position vector.
    \item $A$ is the amplitude.
    \item $\omega$ is the radian frequency.
    \item $\mathbf{k}$ is the wave vector, given by:
    \begin{equation}
        \mathbf{k} = k_x \hat{i}_x + k_z \hat{i}_z
    \end{equation}
    where:
    \begin{align}
        k_x &= k \cos \theta_x \\
        k_z &= k \cos \theta_z
    \end{align}
    \item The wavenumber is given by:
    \begin{equation}
        k = \sqrt{k_x^2 + k_z^2}
    \end{equation}
\end{itemize}

\subsection{Physical Interpretation of the Dot Product $\mathbf{k} \cdot \mathbf{r}$}
By definition, the dot product of two vectors is:
\begin{equation}
    \mathbf{k} \cdot \mathbf{r} = k_x x + k_z z
\end{equation}
which represents the projection of the position vector onto the direction of wave propagation. If we introduce a new coordinate axis $z'$ along the wave vector direction, we can rewrite the wave function as:
\begin{equation}
    \Phi(z',t) = A \cos(\omega t - k z')
\end{equation}
This confirms that the wave propagates along the direction of $\mathbf{k}$.

\section{Example: Dot Product of Two Vectors}
\textbf{Problem:} What is $\mathbf{r} \cdot \hat{i}_x$ equivalent to?

\textbf{Solution:} Since the magnitude of $\hat{i}_x$ is 1 and the projection of $\mathbf{r} = x \hat{i}_x + y \hat{i}_y + z \hat{i}_z$ onto the x-axis is $x$, we get:
\begin{equation}
    \mathbf{r} \cdot \hat{i}_x = x
\end{equation}

\section{Finding the Direction of Propagation and Phase Velocity}
Given the instantaneous expression:
\begin{equation}
    \Phi(x,y,t) = 5 \cos(6 \pi \times 10^9 t - 10^4 x - 10^4 y)
\end{equation}
we determine the direction of propagation as follows:
\begin{itemize}
    \item Comparing with the general form $\Phi(x,y,t) = A \cos(\omega t - k_x x - k_y y)$, we get:
    \begin{align}
        k_x &= 10^4 \\
        k_y &= 10^4
    \end{align}
    \item The propagation direction angle is:
    \begin{equation}
        \theta = \tan^{-1} \left( \frac{k_y}{k_x} \right) = 45^\circ
    \end{equation}
    \item The total wavenumber is:
    \begin{equation}
        k = \sqrt{k_x^2 + k_y^2} = 2 \times 10^4 \text{ rad/m}
    \end{equation}
    \item Given $\omega = 6 \pi \times 10^9$ rad/s, the phase velocity is:
    \begin{equation}
        v_p = \frac{\omega}{k} = 3 \times 10^8 \text{ m/s}
    \end{equation}
    indicating that the wave is an electromagnetic wave propagating in vacuum.
\end{itemize}

\section{Conclusion}
\begin{itemize}
    \item The UPW wave equation can be generalized for arbitrary propagation directions using the wave vector $\mathbf{k}$.
    \item The dot product $\mathbf{k} \cdot \mathbf{r}$ helps determine the phase variation along the propagation direction.
    \item The phase velocity and direction of propagation can be computed using vector components of the wavenumber.
\end{itemize}

\end{document}
