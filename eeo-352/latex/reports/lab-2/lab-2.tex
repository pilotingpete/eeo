\documentclass{article}
\usepackage{blindtext}
\usepackage[letterpaper, total={6.5in, 9in}]{geometry}
\usepackage{tabularray}
\usepackage{hyperref}
\usepackage{xcolor}
\usepackage{amsmath}
\usepackage{siunitx}
\usepackage{graphicx}
\graphicspath{ {./img/} }
\usepackage{pdfpages}

\begin{document}
	
	
	\title{EEO352 Lab 2\\RC Filters and Diodes}
	\author{Pete Mills}
	
	\maketitle
	
	\section*{Copy of Original Assignment}
	
	\includepdf[pages=-]{Assignment 2.pdf}
	
	\section*{1a}
	
	Using the LTspice simulator, design an RC filter with R=1k\(\Omega\) and C=4.7nF 
	
	\begin{center}
	\includegraphics[width=0.7\textwidth]{lpf}
	\end{center}
	
	\subsection*{a}
	
	Simulate and plot the response to a 1V 10kHz sinusoidal signal
	
	\begin{center}
	\includegraphics[width=0.7\textwidth]{1v10k}
	\end{center}
	
	\subsection*{b} 
	
	Simulate and plot the response to a 1V 100kHz sinusoidal signal and extract the phase shift

	
	\begin{center}
	\includegraphics[width=0.7\textwidth]{1v100k}
	\end{center}
	
	
	\[
	\theta = \frac{\Delta T}{T_0}2\pi
	\]

	Where:
	\begin{itemize}
		\item $\theta$ is the phase shift angle in radians.
		\item $\Delta T$ is the time time difference.
	   	\item $T_0$ is the period of the periodic signal.
	  	\item $2\pi$ is the number of radians in one complete cycle.
	\end{itemize}
	
	
	Therefore, the phase shift is $$\theta = \frac{2\mu s}{10\mu s}2\pi = 0.4\pi \text{radians}$$
	
	
	\subsection*{c}
	
	Simulate and plot the frequency response (Bode plot: magnitude and phase)
	
	\begin{center}
	\includegraphics[width=0.7\textwidth]{bode}
	\end{center}
	
	\subsection*{d}
	
	Extract the -3dB frequency and the corresponding phase shift

	\begin{itemize}
		\item From the plot in \textbf{c} above it can be seen the -3db point occurs at $\approx \SI{34}{\kilo\hertz}$
		\item The phase shift is shown to be $\approx \SI{-45}{\degree}$
	\end{itemize}
	
	As a point of interest, I placed the cursor at the $\approx \SI{100}{\kilo\hertz}$ and found the phase shift to be $\SI{-71.3}{\degree}$. This is very close to the $0.4\pi \text{radians}$ value extracted in section \textbf{b} above.
	
	\section*{1b}
	
	Using the Analog Discovery 2 and the components, build and measure the RC filter at (1a) (35 pts):

	
	\begin{center}
	\includegraphics[width=0.7\textwidth]{lpf-setup}
	\end{center}

	\subsection*{a}
	
	Measure and plot the response to a 1V 10kHz sinusoidal signal
	
	The signal is slightly affected by the LPF as can be seen in the oscilloscope screenshot. There is a $\SI{16.4}{\degree}$ phase shift.


	\begin{center}
	\includegraphics[width=0.7\textwidth]{lpf_a1}
	\end{center}
	
	\subsection*{b}
	
	Measure and plot the response to a 1V 100kHz sinusoidal signal and the phase shift
	
	The signal is significantly affected by the LPF as can be seen in the oscilloscope screenshot. There is a $\SI{64.4}{\degree}$ phase shift and the amplitude is reduced to $\SI{601}{\mV}$ p-p. 

	\begin{center}
	\includegraphics[width=0.7\textwidth]{lpf_a2}
	\end{center}
	
	\subsection*{c}
	
	Measure (network function) and plot the frequency response (magnitude and phase)


	Unfortunately I do not have the frequency analysis or function generator modules purchased for this scope.

	Working withing the limitations of my setup, I attempted to perform a manual sweep with the function generator and capture the data on the scope. 
	I exported the scope data to csv and wrote a python program to analyze the data. The program intent was to plot the frequency response of the filter
	in a semilog plot and mark the -3db filter knee. Unfortunately, this method did not produce results I understand at this time.

	
	\begin{center}
	\includegraphics[width=0.7\textwidth]{lpf_manual-sweep}
	\end{center}

	\begin{center}
	\includegraphics[width=0.7\textwidth]{bode-py}
	\end{center}
	
	
	\subsection*{d}
	
	Extrapolate, from the measurement of the resistor and the -3dB frequency, the exact value of the total capacitance
	
	\subsection*{e}
	
	Remove the capacitor and extrapolate, from the measurement of the resistor and the -3dB
	frequency, the value of the residual capacitance from the oscilloscope input
	
	\section*{2a}
	
	Using the LTspice simulator (15pts):
	
	\subsection*{a}
	
	Simulate and plot the diode 1N4148 current for a -1V to +1V diode voltage swing

	\begin{center}
	\includegraphics[width=0.7\textwidth]{diode}
	\end{center}
	
	\subsection*{b}
	
	Place the marker at the 20mA current, report the corresponding voltage and the dynamic resistance (derivative)

	\begin{center}
	\includegraphics[width=0.7\textwidth]{diode-calc}
	\end{center}
	
	To extract the dynamic resistance consider
	
	\[
		R = \frac{\Delta V}{\Delta I}
	\]

	Where:
	\begin{itemize}
		\item $\Delta V$ is the change in voltage.
		\item $\Delta I$ is the change in current.
	\end{itemize}
	
	then the dynamic resistance of the diode under test is 
	 
	\[
		R = \frac{731.31 - 731.22}{20.01 - 19.98} \approx \SI{3.0}{\ohm}
	\]
	
	\section*{2b}
	
	Using the Analog Discovery 2 and the diode 1N4001 (35 pts):


	\begin{center}
	\includegraphics[width=0.7\textwidth]{diode-setup}
	\end{center}

	\begin{center}
	\includegraphics[width=0.7\textwidth]{diode1}
	\end{center}

	\begin{center}
	\includegraphics[width=0.7\textwidth]{diode2}
	\end{center}
	
	\subsection*{a}
	
	Trace the diode current
	
	\begin{itemize}
		\item Use a 100Ω series resistor
		\item Use a +/- 4V 100Hz triangular waveform
	\end{itemize}

	Using x-y mode on the scope I am able to plot the I-V curve of the diode. A schematic of the connections are shown below.
	I used diode 1N4148

	\begin{center}
	\includegraphics[width=0.7\textwidth]{sch}
	\end{center}

	\begin{center}
	\includegraphics[width=0.7\textwidth]{diode3}
	\end{center}


	\subsection*{b}
		
	Zoom to the 20mA current, report the diode voltage, extrapolate the dynamic resistance
	
	\begin{center}
	\includegraphics[width=0.7\textwidth]{diode-dynamic-exp}
	\end{center}
	
	The dynamic resistance $\frac{dV}{dI}$ of the diode under test is calculated as
	 
	\[
		R = \frac{\SI{580}{\mV} - \SI{475}{\mV}}{\SI{40}{\mA} - \SI{10}{\mA}} \approx \SI{3.5}{\ohm}
	\]
	
\end{document}
