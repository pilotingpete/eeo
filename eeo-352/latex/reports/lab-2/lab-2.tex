\documentclass[conference]{IEEEtran}
\usepackage{tabularray}
\usepackage{hyperref}
\usepackage{xcolor}
\usepackage{amsmath}
\usepackage{siunitx}
\usepackage{graphics}

% Define a custom command for text fields with desired properties
\newcommand{\customTextField}[1]{%
	\TextField[bordercolor=red,width=2em]{#1}%
}

\begin{document}
	
	
	\title{EEO352 Lab 2}
	\author{Pete Mills}
	
	% Full-page width cover page
	\IEEEoverridecommandlockouts
	\maketitle
	
	\IEEEpubidadjcol % Adjusts column spacing for cover page
	
	\begin{table}
		\centering
		\caption{Scoring Table}
		
		\begin{tblr}{
				colspec={X[l] X[.2,c] X[.2,c]},
				hlines,
				vlines,
				row{1}={font=\bfseries},
				rowhead=1,
				cells={font=\small}
			}
			\SetCell[c=1]{c}Item & Credit & Score \\
			\SetCell[c=3]{c} Low-pass or High-pass Filter &  & \\
			Circuit Diagram with Values of Components & 1 & \customTextField{} \\
			Measured and Calculated Frequency Plot; -3dB Point & 3 &  \customTextField{} \\
			Transient, Time Response & 1 &  \customTextField{} \\
			\SetCell[c=3]{c} Diodes & & \\
			I-V Plot & 2 &  \customTextField{} \\
			Curve Fitting; IS, $\eta$ & 3 & \customTextField{} \\
			
			Total  & 10 & \customTextField{} \\
		\end{tblr}
	\end{table}
	
	\begin{abstract}
		The abstract is a short summary of the main ideas found in the lab report. It should include 1) the purpose of the study or the question being addressed by the study, 2) the procedures used in the study, 3) the major results of the study, and 4) any conclusions drawn by the author(s).
	\end{abstract}
	
	\section{Circuit Design}
	
	A Low Pass Filter (LPF) shall be designed with a cutoff frequency around 20kHz +/-20\%. 
	
	The equation for a first-order low-pass filter is given by:
	
	\[
	f_c = \frac{1}{2\pi RC} \tag{LPF Cutoff Frequency}
	\]
	
	where \(R\) is resistance, \(C\) is the capacitance, and \(f_c\) is the cutoff frequency.
		
	The equation for resistance (\(R\)) is given by:
	
	\[
	R = \frac{1}{2\pi Cf_c} \tag{LPF Resistance}
	\]
	
	
	
	From the available capacitors in our kit I select 4700pf.
	
	Solving for (\(R\)) we have:
	
	\[
	R = \frac{1}{2\pi 4700pF \cdot 20kHz}
	\]
	
	\[
	R = \SI{1692.58}{\ohm}
	\]
	
	Selecting the nearest standard value (\SI{1.7}{\kohm}) we solve for frequency:
	
	\[
	f_c = \frac{1}{2\pi 1k7 \cdot 4700pF}
	\]
	
	\[
	f_c \approx \SI{19827.51}{\hertz}
	\]
	
	
	The percent difference between \SI{19827.51}{\hertz} and \SI{20000}{\hertz} is approximately
	\[
	\left| \frac{\SI{19827.51}{\hertz} - \SI{20000}{\hertz}}{\frac{\SI{19827.51}{\hertz} + \SI{20000}{\hertz}}{2}} \right| \times 100\% \approx \SI{0.864}{\percent}.
	\]
	
	Which is within the 20\% requirement so we can proceed.
	
	\section{Simulation}
	
	
	
	\section{Data}
	
	\section{analysis}
	
	\section{conclusion}
	
	
\end{document}
