\documentclass[10pt,landscape]{article}
\usepackage{multicol}
%\usepackage{calc}
\usepackage{amssymb}
\usepackage{ifthen}
\usepackage[landscape]{geometry}
\usepackage{graphicx}
%\usepackage{verbatim}
\usepackage{siunitx}



% This sets page margins to .5 inch if using letter paper, and to 1cm
% if using A4 paper. (This probably isn't strictly necessary.)
% If using another size paper, use default 1cm margins.
\ifthenelse{\lengthtest { \paperwidth = 11in}}
	{ \geometry{top=.5in,left=.5in,right=.5in,bottom=.5in} }
	{\ifthenelse{ \lengthtest{ \paperwidth = 297mm}}
		{\geometry{top=1cm,left=1cm,right=1cm,bottom=1cm} }
		{\geometry{top=1cm,left=1cm,right=1cm,bottom=1cm} }
	}

% Turn off header and footer
\pagestyle{empty}
 

% Redefine section commands to use less space
\makeatletter
\renewcommand{\section}{\@startsection{section}{1}{0mm}%
                                {-1ex plus -.5ex minus -.2ex}%
                                {0.5ex plus .2ex}%x
                                {\normalfont\large\bfseries}}
\renewcommand{\subsection}{\@startsection{subsection}{2}{0mm}%
                                {-1explus -.5ex minus -.2ex}%
                                {0.5ex plus .2ex}%
                                {\normalfont\normalsize\bfseries}}
\renewcommand{\subsubsection}{\@startsection{subsubsection}{3}{0mm}%
                                {-1ex plus -.5ex minus -.2ex}%
                                {1ex plus .2ex}%
                                {\normalfont\small\bfseries}}
\makeatother

% Define BibTeX command
\def\BibTeX{{\rm B\kern-.05em{\sc i\kern-.025em b}\kern-.08em
    T\kern-.1667em\lower.7ex\hbox{E}\kern-.125emX}}

% Don't print section numbers
\setcounter{secnumdepth}{0}


\setlength{\parindent}{0pt}
\setlength{\parskip}{0pt plus 0.5ex}


% -----------------------------------------------------------------------

\begin{document}

\raggedright
\footnotesize
\begin{multicols}{3}


% multicol parameters
% These lengths are set only within the two main columns
%\setlength{\columnseprule}{0.25pt}
\setlength{\premulticols}{1pt}
\setlength{\postmulticols}{1pt}
\setlength{\multicolsep}{1pt}
\setlength{\columnsep}{2pt}

% Page title
\begin{center}
     \Large{\textbf{EE Formula Sheet}} \\
\end{center}

\newlength{\MyLen}



%%%%%%%%%%%%%%%%%%_ Begin New Section _%%%%%%%%%%%%%%%%%%%
\section{\S1 - The Crystal Structure of Solids}

\subsection{Miller Index}

\begin{enumerate}
	\item Identify the intercepts: Locate the points where the plane intersects the crystallographic axes (usually denoted as $a$, $b$, and $c$).
	\item Convert to fractional coordinates: Express these intercepts as fractions of the unit cell dimensions ($a$, $b$, $c$). To do this, divide each intercept by the respective unit cell dimension. If an intercept does not intersect the corresponding axis, use $\infty$ as the fraction.
	\item Take reciprocals: Invert the fractional intercepts to obtain the reciprocal fractions.
	\item Simplify the ratios: If any of the reciprocals are not integers, multiply all the indices by the smallest integer that makes them all whole numbers while maintaining the ratio. This ensures that the Miller indices are in the simplest form.
	\item Enclose in parentheses: Write the indices as $(hkl)$, where $h$, $k$, and $l$ are the integers obtained after simplification. These are the Miller indices that represent the crystallographic plane.
	\item Optional: If the plane is parallel to an axis (intercepts at infinity), write the corresponding Miller index as 0. For example, if a plane is parallel to the $a$-axis and intercepts the $b$ and $c$-axes at infinity, its Miller indices would be $(0bc)$.
\end{enumerate}

\subsection{Principles of Quantum Mechanics}

\begin{enumerate}
	\item Superposition Principle: Quantum systems can exist in a linear combination of multiple states simultaneously, described by a wavefunction.
	
	\item Wave-Particle Duality: Particles, such as electrons and photons, exhibit both wave-like and particle-like properties.
	
	\item Uncertainty Principle: There is a fundamental limit to the precision with which certain pairs of properties, such as position and momentum, can be simultaneously known.
	
	\item Quantum States and Operators: Quantum states are described by wavefunctions, and operators represent physical observables and transformations.
	
	\item Measurement and Collapse: Measurement of a quantum system collapses its wavefunction to one of its possible states, with probabilities determined by the square of the amplitude of the wavefunction.
	
	\item Quantum Entanglement: Entangled particles exhibit correlations that cannot be explained by classical physics, even when separated by large distances.
	
	\item Quantum Tunneling: Particles can penetrate energy barriers that classical physics predicts they should not be able to cross.
	
	\item Quantum Interference: Quantum systems can exhibit interference patterns when multiple pathways are available.
	
	\item Quantum Information: Quantum mechanics plays a crucial role in the field of quantum computing and quantum cryptography, offering advantages in information processing and security.
\end{enumerate}

\subsection{Broglie Wavelength}

Convert energy from eV to J:

\[ E = 1.0 \, \text{eV} = 1.60219 \times 10^{-19} \, \text{J} \]

Calculate momentum of the proton or electron:

\[
p_{\text{proton}} = \sqrt{2 \cdot m_{\text{proton}} \cdot E}
\]

Where:
\( p_{\text{proton}} \) = momentum of the proton, 
\( m_{\text{proton}} \) = mass of the proton (\( \approx 1.6726219 \times 10^{-27} \) kg), 
\( E \) = kinetic energy (in Joules)

\[
p_{\text{electron}} = \sqrt{2 \cdot m_{\text{electron}} \cdot E}
\]

Where:
\( p_{\text{electron}} \) = momentum of the electron, 
\( m_{\text{electron}} \) = mass of the electron (\( \approx 9.10938356 \times 10^{-31} \) kg), 
\( E \) = kinetic energy (in Joules)



Calculate the de Broglie wavelength:

\[
\lambda = \frac{h}{p}
\]

Where:
\( \lambda \) = de Broglie wavelength, 
\( h \) = Planck's constant (\(6.62607015 \times 10^{-34} \) m² kg/s)

\subsection{Energy}

The energy (\(E\)) of a photon is given by \(E = h\nu\), where \(h\) is Planck's constant and \(\nu\) is the frequency.

The frequency (\(\nu\)) of a photon is inversely proportional to its wavelength (\(\lambda\)) and can be determined by the equation \(\nu = \frac{c}{\lambda}\), where \(c\) is the speed of light.




%%%%%%%%%%%%%%%%%%_ Begin New Section _%%%%%%%%%%%%%%%%%%%
\section{\S1.2 - The PN Junction}

$V_{bi} = V_T\ln\left(\frac{N_a N_d}{{n^2}_i}\right)$, Built In barrier ($V_f$)\\
$C_j = C_{jo}\left(1+\frac{V_R}{V_{bi}}\right)$, Junction Capacitance\\
$i_D = I_S\left(e^{\left(\frac{v_D}{nV_T}\right)}-1\right)$, Diode Current, where $V_T = 26mV$ @ 300K\\

%%%%%%%%%%%%%%%%%%_ Begin New Section _%%%%%%%%%%%%%%%%%%%
\section{\S1.3 - Diode Circuits: DC Analysis and Models}

\begin{itemize}
	\item Use KVL when $V_D \geq V_y$. Use open circuit when $V_D < V_y$
	\item Use PWL to plot the diode voltage where slope of diode cut in voltage is $m = 1/R_f$
	\item Use KVL formula of the circuit to plot the load line. (Arrange formula in slope-intercept form.)
	\item The \textit{Q-point} is at the intersection of the PWL and load line plots.
\end{itemize}

%%%%%%%%%%%%%%%%%%_ Begin New Section _%%%%%%%%%%%%%%%%%%%
\section{\S1.4 - Diode Circuits: AC Analysis and Models}

First, analyze the DC portion, then the AC portion.

\includegraphics[width = \linewidth]{./img/f1.36.png}

$R_d = \frac{1}{g_d} = \frac{V_T}{I_{DQ}}$, Small Signal Diffusion Resistance\\
$i_d = \left(\frac{I_{DQ}}{V_T}\right) \cdot v_d = g_d \cdot v_d$, AC diode current\\
$v_d = \left(\frac{V_T}{I_{DQ}}\right) \cdot i_d = r_d \cdot i_d$, AC diode voltage\\
where $g_d$ and $r_d$ respectively, are the diode small-signal incremental conductance and resistance, also called the diffusion conductance and diffusion resistance.\\
$C_d = \left(\frac{d_Q}{dV_D}\right)$, Diffusion Capacitance\\

\subsection{Small-Signal Equivalent Circuit}
-\\

%%%%%%%%%%%%%%%%%%_ Begin New Section _%%%%%%%%%%%%%%%%%%%
\section{\S2.1 - Rectifier Circuits}
$\frac{v_i}{v_s} = \frac{N_1}{N_2}$, Transformer voltage to turn-ratio relationship.\\

\subsection{Center tapped formulae}
$v_s(max)) = v_o(max) + V_y$, Peak\\
$v_r(max) = 2v_S(max) - V_y$, Peak Inverse Voltage\\

\subsection{Bridge rectifier formulae}

\includegraphics[width = \linewidth]{./img/f2.9.png}

$v_S(max) = v_O(max)$, Peak\\
$v_R(max) = v_S(max) - V_y$, Peak Inverse Voltage\\
$v_{o^{(t)}} = V_M e^{t'/RC}$, Average Vout\\
$v_L = V_M e^{T'/RC}$, Minimum Vout\\
$v_r = V_M - V_L = \frac{V_M}{2fRC}$, Ripple on Vout\\

%%%%%%%%%%%%%%%%%%_ Begin New Section _%%%%%%%%%%%%%%%%%%%
\section{\S2.4 - Clippers and Clampers}

Clippers clip signals, and clampers shift the entire waveform.

\includegraphics[width = \linewidth]{./img/f2.22.png}
\includegraphics[width = \linewidth]{./img/f2.25.png}
\includegraphics[width = \linewidth]{./img/f2.28.png}

%%%%%%%%%%%%%%%%%%_ Begin New Section _%%%%%%%%%%%%%%%%%%%
\section{\S3.1 - MOS Field-Effect Transistor}


\subsection{N-Channel}

\includegraphics[width = \linewidth]{./img/f3.12.png}

$v_{DS}(sat) = v_{GS} - V_{TN}$, Saturation Voltage, where $V_{TN}$ is the threshold voltage.\\
$i_D = K_n \left[2(v_{GS} - V_{TN} ) v_{DS} - v_{DS}^2\right]$, I-V Characteristic in non-saturation.\\
$i_D = K_n (v_{GS} - V_{TN})^2$, I-V Characteristic in saturation.\\
$C_{ox} = \epsilon_{ox} / t_{ox}$, Oxide capacitance per unit area.\\
$\epsilon_{ox} = (3.9)(\SI{8.85e-14}{F/cm})$, Oxide permittivity for Si devices.\\
$K_n = \frac{W \mu_n C_{ox}}{2L}$, Conduction Parameter\\
$K_n = \frac{k'_n}{2} \cdot \frac{W}{L}$, Conduction Parameter\\
$k'_n = \mu_nC_{ox}$, Process conduction parameter.\\
$\mu_n$, Electron mobility in the inversion layer.\\

\subsection{P-Channel}

\includegraphics[width = \linewidth]{./img/f3.13.png}

$i_D = K_p \left[2(v_{SG} - V_{TP} ) v_{SD} - v_{SD}^2\right]$, I-V Characteristic in non-saturation.\\
$i_D = K_p (v_{SG} - V_{TP})^2$, I-V Characteristic in saturation.\\
$K_p = \frac{W \mu_p C_{ox}}{2L}$, Conduction Parameter\\
$K_p = \frac{k'_p}{2} \cdot \frac{W}{L}$, Conduction Parameter\\
$k'_p = \mu_p C_{ox}$\\

\includegraphics[width = \linewidth]{./img/t3.1.png}

%%%%%%%%%%%%%%%%%%_ Begin New Section _%%%%%%%%%%%%%%%%%%%
\section{\S3.2 - Mosfet DC Analysis}

Establishes the DC operating point, $Q$. This is $I_D$ and $V_{DS}$\\
A resistor on the source leg provides stability via negative feedback at the expense of reducing gain. Alternatively a CC bias may be used to increase stability without limiting gain.

\subsection{Common Source Amplifiers}
\subsubsection{N-Type}
\includegraphics[width = \linewidth]{./img/nmos-common-source.png}

$v_G = v_{GS} = \left(\frac{R2}{R1+R2}\right)V_{DD}$\\
$I_D = K_n(V_{GS} - V_{TN})^2$\\
$V_{DS} = V_{DD} - I_D R_D$\\
$P_T = I_D V_{DS}$, Power\\
If $V_{DS} > V_{DS}(sat)$, where $V_{DS}(sat) = V_{GS} - V_{TN}$, then the transistor is biased in the saturation region\\

\subsubsection{P-Type}
\includegraphics[width = \linewidth]{./img/pmos-common-source.png}

$v_G = \left(\frac{R2}{R1+R2}\right)V_{DD}$\\
$v_{SG} = V_{DD} - V_G$\\
$I_D = K_p(V_{SG} + V_{TP})^2$\\
$V_{SD} = V_{DD} - I_D R_D$\\
$P_T = I_D V_{DS}$, Power\\
If $V_{SD} > V_{SD}(sat)$, where $V_{SD}(sat) = V_{SG} + V_{TP}$, then the transistor is biased in the saturation region\\


\includegraphics[width = \linewidth]{./img/small-signal.png}


%%%%%%%%%%%%%%%%%%_ Begin New Section _%%%%%%%%%%%%%%%%%%%
\section{\S4.1 - Mosfet amplifier}

$g_m = 2\sqrt{K_n I_{DQ}}$, Trans-conductance\\
$r_o = \frac{1}{\lambda I_D}$\\
$r_{eq} = \frac{1}{g_m}$, Small Sig curr source equivalent resistance.\\


%%%%%%%%%%%%%%%%%%_ Begin New Section _%%%%%%%%%%%%%%%%%%%
\section{\S6 - BJT Amplifier}

\includegraphics[width = \linewidth]{./img/6.11.png}

$g_m = 2\sqrt{K_n I_{DQ}}$,\\
$g_m = \frac{I_D}{V_{GS}}$,\\
$g_m = 2K_n (V_{GS}-V_{TH})$,\\
$g_m = \frac{I_C}{V_TH}$,\\
$r_o = \frac{1}{\lambda I_{DQ}}$,\\
$r_o = \frac{V_A}{I_C}$,\\
$r_\pi = \frac{V_T}{I_B}$,\\
$r_\pi = \frac{\beta}{g_m}$,\\
$A_v = -g_m \cdot R_C || R_L$, Voltage Gain Formula\\

%%%%%%%%%%%%%%%%%%_ Begin New Section _%%%%%%%%%%%%%%%%%%%
\subsection{Transistor DC Equivalent}

$V_{th} = \frac{V_{cc}}{R_1 + R_2} \cdot R_2$,\\
$R_{th} = R_1 || R_2$,\\
$V_{ce}(sat) \approxeq 0.2(typ)$,\\
$I_E \approxeq I_C$, In active region\\
$-\frac{1}{R_E-R_C}$, load line slope, where $R_C$ \& $R_E$ are from the AC or DC equivalent circuit. A load line plot is $I_C$ vs $V_{CE}$\\
$I_{RE} = I_B (\beta + 1) R_E$,\\

%%%%%%%%%%%%%%%%%%_ Begin New Section _%%%%%%%%%%%%%%%%%%%
\subsection{Terminology}

\textbf{Common Source:} Input connected to gate, output connected to drain.\\
\textbf{Common Drain (Source Follower):} Input connected to gate, output connected to source.\\
\textbf{Common Gate:} Input connected to source, output connected to drain.\\

\subsection{Transistor formulas}
$I_C = \beta \cdot I_B$, Conduction Parameter\\
$I_B = \frac{I_E}{\beta + 1}$,\\
$\alpha = \frac{I_C}{I_E}$, Current Ratio\\
$I_C = I_E - I_B$, Kirchhoff's Current Law\\
$V_{CE} = V_{BE} + V_{CB}$, Voltage Relationships\\
$I_C = I_{C0} \left( e^{\frac{V_{BE}}{V_T}} - 1 \right)$, BJT Current Equation\\
$I = I_0 \cdot \left( e^{\frac{V}{n \cdot V_T}} - 1 \right)$, Schottky Diode Equation\\
$I_D = \frac{1}{2} \mu_n C_{ox} \frac{W}{L} \left( V_{GS} - V_{TH} \right)^2$, MOSFET Drain Current Equation\\
$I_D = \mu_n C_{ox} \frac{W}{L} \left[ (V_{GS} - V_{TH})V_{DS} - \frac{V_{DS}^2}{2} \right]$, MOSFET Drain Current Equation (Triode Region)\\
$g_m = \sqrt{2 \mu_n C_{ox} \frac{W}{L} I_D}$, Transconductance Parameter\\
$A_v = -g_m \cdot R_D$, Voltage Gain Formula\\

%%%%%%%%%%%%%%%%%%_ Begin New Section _%%%%%%%%%%%%%%%%%%%
\section{EE General Formulae}
$rms = \frac{1}{\sqrt{2}}$,\\
$V = I \cdot R$, Ohm's law.\\
$P = V \cdot I$, DC Power.\\
$P = V \cdot I \cdot \cos(\theta)$, AC power.\\
$E = P \cdot t$, Energy.\\
$C = \frac{Q}{V}$, Capacitance.\\

$V = L \cdot \frac{di}{dt}$, Inductance.\\
$\tau = R \cdot C$, Time constant to reach 63.2\% of capacitors final voltage.\\
$\tau = \frac{L}{R}$, Time constant to reach 63.2\% of inductors final value.
$\frac{N_1}{N_2} = \frac{V_1}{V_2}$, Transformer turns ratio.\\
$V_{\text{peak}} = \sqrt{2} \cdot V_{\text{rms}}$, Peak AC Voltage.\\
$V_{\text{rms}} = \frac{V_{\text{peak}}}{\sqrt{2}}$, RMS AC Voltage.\\
$V_{\text{avg}} = \frac{1}{T} \int_0^T V(t) \, dt$, RMS AC Voltage.\\
$V_{\text{out}} = V_{\text{in}} \cdot \frac{R_2}{R_1 + R_2}$, voltage divider.\\
$R_{\text{eq}} = R_1 + R_2 + \ldots + R_n$, series resistors.\\
$\frac{1}{R_{\text{eq}}} = \frac{1}{R_1} + \frac{1}{R_2} + \ldots + \frac{1}{R_n}$, Parallel resistors.\\
$\frac{1}{C_{\text{eq}}} = \frac{1}{C_1} + \frac{1}{C_2} + \ldots + \frac{1}{C_n}$, Series capacitors.\\
$C_{\text{eq}} = C_1 + C_2 + \ldots + C_n$, parallel capacitors.\\


%%%%%%%%%%%%%%%%%%_ Begin New Section _%%%%%%%%%%%%%%%%%%%
\section{Convert Polar to Rectangular}

$$x = r \cos \theta$$
$$y = r \sin \theta$$

\section{Exact Slope of a Tangent Line}

$$\frac{dy}{dx} = \frac{dy/dt}{dx/dt}$$





%%%%%%%%%%%%%%%%%%_ Begin New Section _%%%%%%%%%%%%%%%%%%%
\section{Basic integration Rules}

$\int kf(u) du = k \int f(u) du + C$, $\int [f(u) \pm g(u)]du = \int f(u) du \pm \int g(u) du$, 
$\int du = u +C$, $\int u^n du = \frac{u^{n+1}}{n+1} +C, n \neq -1$, $\int \frac{du}{u} = \ln |u| +C$, $\int \frac{u}{du} = \frac{u^2}{2} +C$,
$\int e^u du = e^u +C$,$\int e^{4u} = \frac{e^{4u}}{4} +C$, $\int a^u du = \left( \frac{1}{\ln a} \right) a^u +C$, \\

\subsection{Some Integrals}

$\int \sin u du = -\cos u+C$, $\int \cos u du = \sin u +C$,
$\int \tan u du = -\ln| \cos u | +C$, $\int \cot u du = \ln| \sin u | +C$, $\int \sec u du = \ln| \sec u + \tan u | +C$,
$\int \csc u du = -\ln| \csc u + \cot u | +C$, $\int \sec^2 u du = \tan u +C$, $\int \csc^2 u du = -\cot u +C$,
$\int \sec u \tan u du = \sec u +C$, $\int \csc u \cot u du = -\csc u +C$, $\int \frac{du}{\sqrt{a^2-u^2}} = \arcsin \frac{u}{a} +C$,
$\int \frac{du}{a^2+u^2} = \frac{1}{a} \arctan \frac{u}{a} +C$,
$\int \frac{du}{u\sqrt{u^2-a^2}} = \frac{1}{a} $arcsec$ \frac{|u|}{a} +C$, $\int \sin 3x = -\frac{1}{3} \cos 3x$, $\int e^{-4x} = \frac{e^{-4x}}{-4}$

$\int k dx = kx+C$, $\int x dx = \frac{1}{2}x^2 +C$, $\int x^2 dx = \frac{1}{3} x^3 +C$, $\int \frac{1}{x} dx = \ln|x| + C$,
$\int e^x dx = e^x +C$,$\int k^u du = \frac{k^u}{\ln u} +C$,  $\int \ln x dx = x \ln x - x +C$, $\int \cos x dx = \sin x +C$,
$\int \sin x dx = -\cos x +C$, $\int \sec^2 x dx = \tan x +C$, $\int x^n dx = \frac{x^{n+1}}{n+1} +C$, $\int \tan x = - \ln( \cos x ) +C$,


%%%%%%%%%%%%%%%%%%_ Begin New Section _%%%%%%%%%%%%%%%%%%%
\subsection{Integration by Parts}

$$\int u dv = uv - \int v du$$




%%%%%%%%%%%%%%%%%%_ Begin New Section _%%%%%%%%%%%%%%%%%%%
\section{Some Identities}

$\sin2x = 2 \sin x \cos x$

\subsection{Pythagorean:}
$\sin^2 x+ \cos^2 x = 1$, $1 + \tan^2 x = \sec^2x$, $1 + cot^2 x = \csc^2 x$ 

\subsection{Reciprocal:}
$\sin x = \frac{1}{\csc x}$, $\cos x = \frac{1}{\sec x}$, $\tan x = \frac{\sin x}{\cos x} = \frac{1}{\cot x}$\\
$\csc x = \frac{1}{\sin x}$, $\sec x = \frac{1}{\cos x}$, $\cot x = \frac{\cos x}{\sin x} = \frac{1}{\tan x}$ 

\subsection{Half Angle:}
$\sin^2 x = \frac{1}{2} (1 - \cos 2x)$, $\cos^2 x = \frac{1}{2} ( 1 + \cos 2x)$

%%%%%%%%%%%%%%%%%%_ Begin New Section _%%%%%%%%%%%%%%%%%%%
\section{Additional Notes:}


$\ln ( x * y ) = \ln(x) + \ln(y)$, $\ln ( x / y ) = \ln( x ) - \ln ( y )$

$\ln x^a = a \ln x$, $\tan \theta = \frac{\sin \theta}{\cos \theta}$\\

$ax^2+bx+c = 0$, $x= \frac{-b \pm \sqrt{b^2-4ac}}{2a}$\\

$\ln a = c \equiv e^c = a$\\

$\sqrt[n]{a} = a^\frac{1}{n}$, $a^{-n} = \frac{1}{a^n}$, $\sqrt[n]{a^m} = a^\frac{m}{n}$, $a^0 = 1$, $\left( a^m \right)^n = a^{mn}$, 
$a^m * a^n = a^{m + n}$, $\frac{ a^m }{ a^n } = a^{m-n}$, Rewrite $\sqrt{5x}$ as $\sqrt{5} \sqrt{x}$, \\





\section{Some Derivatives:}

$\frac{d}{du} \sin u = (\cos u)u'$, $\frac{d}{du} \cos u = -(\sin u)u'$, $\frac{d}{du} \tan u = (\sec^2 u)u'$,
$\frac{d}{du} \cot u = -(\csc^2 u)u'$, $\frac{d}{du} \sec u = (\sec u \tan u)u'$, $\frac{d}{du} \csc u = -(\csc u \cot u)u'$,
$\frac{d}{du} \arcsin u = \frac{u'}{\sqrt{1 - u^2}}$, $\frac{d}{du} \arccos u = \frac{-u'}{\sqrt{1 - u^2}}$,
$\frac{d}{du} \arctan u = \frac{u'}{1 + u^2}$, $\frac{d}{du} $arccot $ u = \frac{-u'}{1 + u^2}$, \\
$\frac{d}{du} $arcsec $ u = \frac{u'}{|u|\sqrt{u^2 - 1}}$,
$\frac{d}{du} $arccsc $ u = \frac{-u'}{|u|\sqrt{u^2 - 1}}$

$\frac{d}{du}[\ln{u}] = \frac{1}{u}u'$, $\frac{d}{dx}[e^{-x}] = -e^{-x}$, $e^{\ln a} = a$ \\
$\frac{d}{du}[\sqrt{u}] = \frac{u'}{2 \sqrt{u}}$, $e^{3x} = 3e^{3x}$, $\frac{d}{dx}\left[ x \right] = 1$, $\frac{d}{dx}\left[ c \right] = 0$, 
$\frac{d}{du}[ \frac{1}{u} ] = \frac{1}{u^2}$, $\frac{du}{u} = \ln |u|$,


\includegraphics[width = \linewidth]{./img/unitCircle}






\end{multicols}
\end{document}
