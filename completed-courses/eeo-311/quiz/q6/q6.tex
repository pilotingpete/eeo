% !TEX TS-program = pdflatex
% !TEX encoding = UTF-8 Unicode

% This is a simple template for a LaTeX document using the "article" class.
% See "book", "report", "letter" for other types of document.

\documentclass[11pt]{article} % use larger type; default would be 10pt

\usepackage[utf8]{inputenc} % set input encoding (not needed with XeLaTeX)

\usepackage{siunitx}

%%% Examples of Article customizations
% These packages are optional, depending whether you want the features they provide.
% See the LaTeX Companion or other references for full information.

%%% PAGE DIMENSIONS
\usepackage{geometry} % to change the page dimensions
\geometry{a4paper} % or letterpaper (US) or a5paper or....
% \geometry{margin=2in} % for example, change the margins to 2 inches all round
% \geometry{landscape} % set up the page for landscape
%   read geometry.pdf for detailed page layout information

\usepackage{graphicx} % support the \includegraphics command and options

% \usepackage[parfill]{parskip} % Activate to begin paragraphs with an empty line rather than an indent

%%% PACKAGES
\usepackage{booktabs} % for much better looking tables
\usepackage{array} % for better arrays (eg matrices) in maths
\usepackage{paralist} % very flexible & customisable lists (eg. enumerate/itemize, etc.)
\usepackage{verbatim} % adds environment for commenting out blocks of text & for better verbatim
\usepackage{subfig} % make it possible to include more than one captioned figure/table in a single float
% These packages are all incorporated in the memoir class to one degree or another...

%%% HEADERS & FOOTERS
\usepackage{fancyhdr} % This should be set AFTER setting up the page geometry
\pagestyle{fancy} % options: empty , plain , fancy
\renewcommand{\headrulewidth}{0pt} % customise the layout...
\lhead{}\chead{}\rhead{}
\lfoot{}\cfoot{\thepage}\rfoot{}

%%% SECTION TITLE APPEARANCE
\usepackage{sectsty}
\allsectionsfont{\sffamily\mdseries\upshape} % (See the fntguide.pdf for font help)
% (This matches ConTeXt defaults)

%%% ToC (table of contents) APPEARANCE
\usepackage[nottoc,notlof,notlot]{tocbibind} % Put the bibliography in the ToC
\usepackage[titles,subfigure]{tocloft} % Alter the style of the Table of Contents
\renewcommand{\cftsecfont}{\rmfamily\mdseries\upshape}
\renewcommand{\cftsecpagefont}{\rmfamily\mdseries\upshape} % No bold!

%%% END Article customizations

%%% The "real" document content comes below...

\title{EEO 311 \\ Quiz 6}
\author{Pete Mills}
%\date{} % Activate to display a given date or no date (if empty),
         % otherwise the current date is printed 

\begin{document}
\maketitle

\section{What would be the DC gain  of the  2-stage circuit in dB units?}

Find total gain $AV_{total}$

$$AV_{total} = AV_1 \cdot AV_2$$
$$AV_{total} = 10 \cdot 10 = 100$$
$$AV_{total} = 20\log(100) = \SI{40}\decibel$$

\section{A sinusoidal voltage with the amplitude of 1 mV and a 100 MHz frequency was applied to the input of the 1st stage of the circuit. What would be the amplitude of the output voltage of the 2-stage circuit?}

Find $V_{out}$

$$V_{out} = V_{in} \cdot (\frac{1}{\sqrt{2}} \cdot AV_1) \cdot (\frac{1}{\sqrt{2}} \cdot AV_2)$$

$$V_{out} = \SI{1}{\mV} \cdot (\frac{1}{2} \cdot 100) = \SI{50}{mV}$$

\section{What would be the cut-off frequency of the 2-stage circuit in MHz at the -3 dB level?}


\[
f_{c\_2\text{-stage}} = \frac{1}{\sqrt{2}} \times f = 0.707 \times 100 \, \text{MHz} = 70.7 \, \text{MHz}
\]

\section{What would be the phase angle of the transfer function of the 2-stage circuit at the cut-off frequency? }

Phase shifts are summed in series-connected LPF's. Each Stage is introducing $\SI{-45}\degree$, therefore a total phase shift of $\SI{-90}\degree$ is seen at the output.


























































\end{document}
