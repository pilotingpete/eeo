\documentclass{article}
%\usepackage{blindtext}
\usepackage[letterpaper, total={6.5in, 9in}]{geometry}
%\usepackage{tabularray}
%\usepackage{hyperref}
%\usepackage{xcolor}
\usepackage{amsmath}
\usepackage{siunitx}
\usepackage{graphicx}
\graphicspath{ {./img/} }
\usepackage{pdfpages}
\usepackage{float}
\usepackage[normalem]{ulem}


\begin{document}
	
\begin{titlepage}
	\centering
	\includegraphics[width=0.45\textwidth]{sbu}\par\vspace{1cm}
	{\LARGE \textsc{EEO335}\par}
	\vspace{1cm}
	{\Large \textsc{Spring 2024}\par}
	\vspace{1.5cm}
	{\huge\bfseries Infrared Transmitter and Receiver\par}
	\vspace{2cm}
	{\Large\itshape Pete Mills\\ID: 115009163\par}
	\vfill
	Professor\par
	Gianluigi \textsc{De Geronimo}

	\vfill

% Bottom of the page
	{\large \today\par}
\end{titlepage}

% Include the original assignment
	\newcommand{\assName}{Assignment 6-1.pdf}

	\includepdf[pages=-,pagecommand=\section*{Copy of Original Assignment}]{\assName}
	%\includepdf[pages=2-,pagecommand={}]{\assName}
	
	%\listoffigures
	
% Begin report

	\section*{Overview}
	
	 In this lab we explore the performance of optocouplers through experiments and simulations. We first design and simulate a circuit, then build and test it practically using specified components. Then, we measure responses, calculate gains, and note observations. This assignment aims to reinforce understanding while providing hands-on experience with infrared transmission systems.

	
	\section{Using the simulator, design the circuit shown in Fig. 1}

	\subsection*{a)}
	
	\begin{figure}[H]
	    \centering
	    \includegraphics[width=0.7\textwidth]{1a}
	    \caption{The simulation reveals an output gain of $\approx 43.7$}
	\end{figure}

	\subsection*{b)}
	
	This circuit is an optically coupled transmitter/receiver circuit. There is no electrical continuity between the input and the output, yet a signal can still be transmitted.
	
	 First, I will describe the transmitter portion of the circuit. Capacitor C1 removes any DC voltage on the input signal and passes only AC signal. R1 and R2 are equal and thus provide a balanced bias to the input of Q2. R3 limits current in the IR LED transmitter of U1 and C2 filters the signal at the emitter of Q2. 
	
	For the receiver portion of the circuit, the IR light emitted by the transmitter in U1 falls on the `base' of the receiver/NPN output of U1. In this way you can see that the IR light modulated from the transmitter to receiver can transmit a signal. This output is fed to U2 which is configured as an inverting amplifier configuration. With R4 and R5 being equal, it is understood the gain in U2 will be $-1$.
	
	\section{Build the circuit at (1) and experimentally reproduce the simulation}    

	\subsection*{a)}
	
	\begin{figure}[H]
	    \centering
	    \includegraphics[width=0.7\textwidth]{c1}
	    \caption{Experimental circuit with IR transmitter and receiver placed head-to-head.}
	\end{figure}
	
	\begin{figure}[H]
	    \centering
	    \includegraphics[width=0.7\textwidth]{w1}
	    \caption{Experimentation reveals a system gain of $\approx 18$. This is lower than the simulation, however the components in the simulation are also different to the ones used in the experiement and have different parameters leading to different gain.}
	\end{figure}
	
	\subsection*{b)}
	
	\begin{figure}[H]
	    \centering
	    \includegraphics[width=0.7\textwidth]{w2}
	    \caption{Noticeable output distortion occurs when the input signal reaches just $\SI{13}{\mV}$}
	\end{figure}

\end{document}
