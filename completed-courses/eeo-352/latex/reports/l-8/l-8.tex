\documentclass{article}
%\usepackage{blindtext}
\usepackage[letterpaper, total={6.5in, 9in}]{geometry}
%\usepackage{tabularray}
%\usepackage{hyperref}
%\usepackage{xcolor}
\usepackage{amsmath}
\usepackage{siunitx}
\usepackage{graphicx}
\graphicspath{ {./img/} }
\usepackage{pdfpages}
\usepackage{float}



\begin{document}
	
	
	\title{EEO352 Lab 8\\ Bipolar Junction Transistors (BJTs)}
	\author{Pete Mills}
	
	\maketitle
	
	\section*{Copy of Original Assignment}
	
	\includepdf[pages=-]{Assignment 8.pdf}
	
	%\listoffigures
	
	\section*{Summary}
	
	In this lab we ...
	

	\section{Simulate and Measure Circuit 1}
	
	\subsection*{a)}

	\begin{figure}[H]
	    \centering
	    \includegraphics[width=0.7\textwidth]{1a}
	    \caption{Collector current Ic vs Vce (range 0V to 3V) with parametric Vbe (0.5V to 0.75V, 10mV steps)}
	\end{figure}
	
	\subsection*{b)}

	\begin{figure}[H]
	    \centering
	    \includegraphics[width=0.7\textwidth]{1b}
	    \caption{Collector current Ic vs Vbe (range 0.5V to 0.75V) with parametric Vce (0.5V to 3V in 100mV steps)}
	\end{figure}
	
	\subsection*{c)}

	\begin{figure}[H]
	    \centering
	    \includegraphics[width=0.7\textwidth]{1c}
	    \caption{Collector current Ic vs Vbe (range 0.5V to 0.75V) at Vce=2V, and extrapolate the Vbe at Ic=1mA}
	\end{figure}
	
	$V_{be}$ is $\approx$ \SI{655}{\milli\volt} @ $I_c = $\SI{1}{\mA}
	
	\subsection*{d)}

	\begin{figure}[H]
	    \centering
	    \includegraphics[width=0.7\textwidth]{1d}
	    \caption{Derivative (d(.)) of the collector current Ic vs Vbe (range 0.5V to 0.75V) at Vce=2V, and extrapolate the
transconductance at Ic=1mA}
	\end{figure}
	
	The transconductance $g_m$ is found to be \SI{39.042}{\milli\ohm} @ $I_c$ = \SI{1}{\mA}
	
	\subsection*{e)}

	\begin{figure}[H]
	    \centering
	    \includegraphics[width=0.7\textwidth]{1e}
	    \caption{Current gain Ic/Ib vs Vbe (range 0.5V to 0.75V) at Vce=2V, and extrapolate the gain at Ic=1mA}
	\end{figure}
	
	The gain is $\approx 303.3$

	
	
	
	
	\section{Simulate and Measure Circuit 2}
	
	\subsection*{a)}

	\begin{figure}[H]
	    \centering
	    \includegraphics[width=0.7\textwidth]{2a}
	    \caption{Response to 50kHz 5mV sinusoidal signal (plot in separate panes) and extrapolate the gain}
	\end{figure}
	
	The input amplitude is \SI{5}{\mV} pp and output amplitude is \SI{1.179}{\volt} pp, centered about \SI{1.77}{\volt}, therefore the gain is $\approx 236$.
	
	\subsection*{b)}

	\begin{figure}[H]
	    \centering
	    \includegraphics[width=0.7\textwidth]{2b-1}
	    \includegraphics[width=0.7\textwidth]{2b-2}
	    \caption{Frequency response, extrapolating the gain and -3dB bandwidth without and with a 30pF load}
	\end{figure}
	
	Without the \SI{30}{\pico\farad} output capacitance, the \SI{-3}{\decibel} point is found at \SI{1.0}{\giga\hertz}\\
	With the \SI{30}{\pico\farad} output capacitance, the \SI{-3}{\decibel} point is found at \SI{259.2}{\mega\hertz}	
	
	

	\section{Build and Measure Circuit 1}
	
	\subsection*{a)}
	
	\begin{figure}[H]
	    \centering
	    \includegraphics[width=0.7\textwidth]{3a-photo}
	    \caption{Test Setup}
	\end{figure}
	
	\begin{figure}[H]
	    \centering
	    \includegraphics[width=0.7\textwidth]{3a}
	    \caption{Collector current Ic vs Vce (range 0V to 3V) with parametric Vbe (see example in Fig.3)}
	\end{figure}
	
	\subsection*{b)}
	
	\begin{figure}[H]
	    \centering
	    \includegraphics[width=0.7\textwidth]{3b-photo}
	    \caption{Test Setup}
	\end{figure}
	
	\begin{figure}[H]
	    \centering
	    \includegraphics[width=0.7\textwidth]{3b}
	    \caption{Collector current Ic vs Vbe (range 0.5 to 0.75V) for Vce>2V, and extract the Vbe and the gm at Ic=1mA}
	\end{figure}
	
	$V_{be}$ is $\approx$ \SI{664}{\milli\volt} @ $I_c = $\SI{1}{\mA}


	\begin{figure}[H]
	    \centering
	    \includegraphics[width=0.7\textwidth]{3b-2}
	    \caption{Zooming in to calculate $gm = \Delta V / \Delta I$}
	\end{figure}
	
	At $I_c = \SI{1}{\mA}$, $gm \approx \SI{25}{\milli\Omega}$
	
	

	\subsection*{c)}
	
	\begin{figure}[H]
	    \centering
	    \includegraphics[width=0.7\textwidth]{3c}
	    \caption{Gain Ic/Ib vs Collector current Ic (setting as in (b)) and extract the gain at Ic=1mA}
	\end{figure}
	
	The gain is found to be $\approx 180$
	
	
	
	\section{Build and Measure Circuit 2}
	
	\subsection*{a)}
	
	\begin{figure}[H]
	    \centering
	    \includegraphics[width=0.7\textwidth]{4a-photo}
	    \caption{Test Setup}
	\end{figure}
	
	\begin{figure}[H]
	    \centering
	    \includegraphics[width=0.7\textwidth]{4a}
	    \caption{Response to 50kHz 5mV sinusoidal signal and extract the gain}
	\end{figure}
	
	The input amplitude is \SI{5}{\mV} pp and output amplitude is \SI{0.57}{\volt} pp, centered about \SI{2.4}{\volt}, therefore the gain is 114.
	
	\subsection*{b)}
	
	\begin{figure}[H]
	    \centering
	    \includegraphics[width=0.7\textwidth]{4b}
	    \caption{Frequency response, extracting the gain and the -3dB bandwidth}
	\end{figure}
	
	 Using the Digilent Waveforms Network Analyzer it appears there is a \SI{-3}{\decibel} point at about \SI{900}{\hertz} and another at at about \SI{900}{\kilo\hertz}. At the time of writing, I am not sure what this means, but will continue to research to understand.
	
	
	
		
\end{document}
